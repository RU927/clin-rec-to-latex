\section{Лечение, включая медикаментозную и немедикаментозную терапии, диетотерапию, обезболивание, медицинские показания и противопоказания к применению методов лечения}
\label{sec:Treatment}
Выбор метода лечения уротелиального рака, прежде всего, определяется наличием метастазов, глубиной инвазии первичной опухоли, ее грейдом и сопутствующей CIS. Важными факторами, влияющими на лечебную тактику, являются локализация и количество опухолевых очагов. При индивидуальном выборе метода лечения больных уротелиальным раком также принимаются во внимание функциональная сохранность пораженного участка мочевыводящих путей, коморбидный фон и ожидаемая продолжительность жизни.

Клинически уротелиальный рак можно разделить на 3 категории, которые принципиально различаются по прогнозу, видам и целями лечения. К первой категории относятся немышечно-инвазивные опухоли, лечение которых направлено на радикальное удаление новообразования, снижение риска рецидива и предотвращение опухолевой прогрессии в мышечно-инвазивный уротелиальный рак.

Вторая группа включает мышечно-инвазивные уротелиальные карциномы. Целью их лечения является радикальное удаление опухоли, снижение риска метастазирования и поддержание качества жизни за счет сохранения пораженного органа. Органосохраняющее лечение возможно только в случаях, когда это не приведет к снижению выживаемости. В зависимости от индивидуального риска диссеминации мышечно-инвазивного уротелиального рака принимается решение о необходимости назначения системной противоопухолевой терапии.

К третьей группе уротелиальных раков относятся неоперабельные местно-распространенные и диссеминированные формы заболевания. Целью лечения этой категории больных является увеличение продолжительности и сохранение качества жизни за счет последовательного применения противоопухолевых препаратов различного механизма действия и их комбинаций.


\subsection{Лечение немышечно-инвазивного рака мочевого пузыря}
\label{sec:}
3.1.1.Трансуретральная резекция
Рекомендуется начинать лечение немышечно-инвазивного рака мочевого пузыря (НМИ РМП) с ТУР мочевого пузыря (за исключением пациентов с тотальным поражением МП - таким пациентам показана ЦЭ). [104, 117].

Уровень убедительности рекомендаций – С (уровень достоверности доказательств – 5).

Комментарии: При ТУР МП удаляют все видимые опухоли. Отдельно удаляют экзофитный компонент и основание опухоли. Это необходимо для правильного установления стадии заболевания (рТ), так как в зависимости от полученных результатов вырабатывают дальнейшую тактику лечения пациента. Проведение ТУР МП с последующим патоморфологическим исследованием – главный этап в лечении НМИ РМП. Целью лечения в данном случае является удаление существующей опухоли с профилактикой рецидива заболевания и предотвращением развития инфильтративной опухоли.

Наиболее распространенными осложнениями ТУР МП являются:

➢ кровотечения (интраоперационные и послеоперационные), иногда требующие открытого хирургического вмешательства;

➢ перфорация стенки мочевого пузыря (внутрибрюшинная перфорация требует лапаротомии или лапароскопии, дренирования брюшной полости, ушивания дефекта стенки мочевого пузыря).

Рекомендуется выполнять повторную ТУР (second-look) для верификации диагноза в следующих случаях:

− после неполной первоначальной ТУР – для исключения опухолей TaG1 и первичного РМП, если после первоначальной резекции в образце не было мышечной ткани;

− во всех случаях опухолей Т1;

− при всех опухолях G3 [118–122].

Уровень убедительности рекомендаций – С (уровень достоверности доказательств – 4).

Комментарии: Проведение повторной ТУР является обязательной манипуляцией у пациентов группы высокого риска. Исследования демонстрируют достоверные различия в безрецидивной выживаемости и выживаемости без прогрессии [118–121]. Повторная ТУР выполняется через 2–6 недель после первичной процедуры [122].

При некоторых экзофитных опухолях возможна резекция единым блоком (en bloc) с использованием моно- или биполярного тока, а также современных методов: лазеров (тулиевый и гольмиевый) Такая методика обеспечивает высокое качество морфологического материала с наличием мышечного слоя в 96–100\% случаев [123–126].

3.1.2. Тактика ведения пациентов с немышечно-инвазивным раком мочевого пузыря после трансуретральной резекции
3.1.2.1. Однократная немедленная внутрипузырная инстилляция химиопрепарата
При использовании ТУР можно полностью удалить макроопухоль, но невозможно повлиять на микроочаги. В результате возникают рецидивы, которые могут в дальнейшем прогрессировать до МИ РМП [117]. Поэтому необходимо рассмотреть вопрос об адъювантной терапии у всех пациентов [211]. 

Рекомендуется однократная немедленная (в первые 6 часов после ТУР) внутрипузырная инстилляция химиопрепарата (противоопухолевого антибиотика или родственного соединения) всем пациентам с НМИ РМП вне зависимости от группы риска для снижения частоты развития рецидивов [211, 212].

Уровень убедительности рекомендаций – А (уровень достоверности доказательств – 1).

Комментарии: При лечении пациентов с НМИ РМП с высокой вероятностью развития рецидива в первые 3 мес. наблюдения рекомендуется рассматривать назначение адъювантной терапии. Применение внутрипузырной химиотерапии приводит к снижению рецидивов, увеличению продолжительности безрецидивного течения, однако не сказывается на частоте прогрессирования процесса и показателях выживаемости [212]. 

Ранняя послеоперационная инстилляция не проводится в случаях явной или предполагаемой перфорации стенки мочевого пузыря, а также при гематурии, когда требуется промывание полости МП. В данном случае среднему медицинскому персоналу необходимо давать четкие инструкции по контролю свободного оттока жидкости по мочевому катетеру. Необходимость в проведении адъювантной внутрипузырной терапии зависит от прогноза рецидива заболевания[213].  

В группе пациентов низкого риска немедленная однократная химиотерапия проводится в качестве полной (завершенной) адъювантной терапии. Данной категории пациентов не требуется лечения до последующего рецидива [214]. Однако для других групп риска однократная немедленная инстилляция является недостаточной из-за высокой вероятности развития рецидива и/или прогрессирования. 

Рекомендуется проводить внутрипузырную экспозицию митомицином** или доксорубицином** всем пациентам с НМИ РМП в течение 1 часа для минимизации побочных эффектов [254, 306-308].

Уровень убедительности рекомендаций – В (уровень достоверности доказательств – 2).

Комментарии: Длительность экспозиции химиопрепарата также регламентирована. При сравнении 0,5 и 1-часовой экспозиций достоверной разницы в безрецидивной выживаемости не отмечено [254].

3.1.2.2. Адъювантная внутрипузырная терапия
Выбор тактики дальнейшего лечения и наблюдения определяетя на основании таблиц и номограмм, предложенных Европейской ассоциацией по изучению и лечению рака в 2006г. [127]. В зависимости от прогностических факторов возникновения рецидива и прогрессии у пациентов с НМИ РМП рекомендована выработка дальнейшей тактики лечения [127].

Рекомендуется проведение цистоскопии пациентам с НМИ РМП группы низкого риска после выполнения ТУР и однократной инстилляции химиопрепарата из группы противоопухолевых антибиотиков (доксорубицин**, митомицин**) с целью динамического наблюдения [127, 281].

Уровень убедительности рекомендаций – С (уровень достоверности доказательств – 2).

Комментарии: группа низкого риска – уровень инвазии рТа, дифференцировка G1, единичная опухоль менее 3 см, отсутствие CIS. Риск рецидива и прогрессирования опухоли в данной группе за 5 лет – до 37 и 1,7 \% соответственно. Смертность за 10 лет – 4,3 \%.

Рекомендуется проведение адъювантной внутрипузырной терапии вакциной для иммунотерапии рака мочевого пузыря** или химиотерапии противоопухолевым антибиотиком (доксорубицин**, митомицин**) пациентам с НМИ РМП группы промежуточного риска после выполнения ТУР и однократной инстилляции противоопухолевого антибиотика (доксорубицин**, митомицин**) с целью снижения риска рецидивов [127, 205].

Уровень убедительности рекомендаций – B (уровень достоверности доказательств – 2).

Комментарии: к этой группе относятся все пациенты, не вошедшие в группу низкого или высокого риска. Риск рецидива и прогрессирования опухоли за 5 лет – до 65 и 8 \% соответственно. Смертность за 10 лет – 12,8 \%.

Рекомендуется назначение адъювантной терапии всем пациентам с НМИ РМП группы высокого риска. Предпочтение стоит отдавать БЦЖ-терапии с поддерживающим режимом [127, 204, 205].

Уровень убедительности рекомендаций – В (уровень достоверности доказательств – 2).

Комментарии: группа высокого риска – уровень инвазии рТ1, дифференцировка G3, множественные и рецидивные опухоли; CIS, а также большие опухоли (более 3 см), pTaG1–2 при возникновении рецидива в течение 6 мес. после операции. Эта группа прогностически неблагоприятная. Эффективность внутрипузырной химиотерапии значительно ниже. Вариант выбора у данных пациентов при неэффективности комбинированного органосохраняющего лечения – ЦЭ. Риск рецидива и прогрессирования опухоли за 5 лет – до 84 и 55 \% соответственно. Смертность за 10 лет – 36,1 \%. Индукционные инстилляции вакцина для иммунотерапии рака мочевого пузыря** классически выполняются в соответствии с эмпирической 6-недельной схемой, которая была предложена Morales и соавт. [200].

Рекомендуется проведение внутрипузырной БЦЖ-терапии с использованием полной дозы в течение 1–3 лет пациентам с НМИ РМП групп промежуточного и высокого риска развития рецидива и прогрессирования для достижения ремиссии [200-204].

Уровень убедительности рекомендаций – А (уровень достоверности доказательств – 2).

Комментарии: в мета-анализе положительный эффект наблюдался только у пациентов, получивших БЦЖ-терапию по поддерживающей схеме. Используется много различных поддерживающих режимов: от 10 инстилляций, проведенных в течение 18 недель, до 27 более чем за 3 года. С помощью мета-анализа невозможно было определить, какая поддерживающая схема вакцины была наиболее эффективной. Преимущество иммунотерапии перед митомицином** в предупреждении развития рецидива и прогрессирования появляется только при применении БЦЖ-терапии продолжительностью не менее 1 года. Оптимальное количество, частота и длительность поддерживающих индукционных инстилляций остаются неизвестными. Однако результаты рандомизированного контролируемого исследования, куда вошли 1355 пациентов, показали, что проведение поддерживающей БЦЖ-терапии в течение 3 лет с использованием полной дозы вакцины снижает частоту рецидивирования по сравнению с 1 годом лечения в группе высокого риска, но это не относится к пациентам с промежуточным риском. Не наблюдалось различий при сравнении показателей прогрессирования или общей выживаемости [200–204].  

Рекомендуется пациентам с опухолью в простатической части уретры выполнение ТУР предстательной железы с последующими внутрипузырными инстилляциями вакциной для иммунотерапии рака мочевого пузыря** с целью снижения частоты рецидивов [84, 202].

Уровень убедительности рекомендаций – С (уровень достоверности доказательств – 5).

Комментарии: первые инстилляции проводятся через 3–4 нед. после ТУР. Вакцина для иммунотерапии рака мочевого пузыря**: 50–100 мг в 50мл физиологического раствора натрия хлорида**. Вводится еженедельно, в течение 6 нед, далее - ежемесячно на протяжении 1 года, либо по схеме: 3 недельные циклы каждые 3, 6, 12, 18, 24, 30, 36 мес. При БЦЖ-рефрактерных опухолях целесообразно выполнение радикальной ЦЭ.

Не рекомендуется проведение внутрипузырной инстилляции вакцины для иммунотерапии рака мочевого пузыря** в следующих случаях [205, 206]: 

в течение первых 2 недель после ТУР; 

пациентам с макрогематурией; 

после травматичной катетеризации; 

пациентам с наличием симптомов ИМП. 

Уровень убедительности рекомендаций – С (уровень достоверности доказательств – 5).

Комментарии: наличие лейкоцитурии или асимптоматической бактериурии не является противопоказанием для проведения БЦЖ-терапии, в этих случаях нет необходимости в проведении антибиотикопрофилактики. Системные осложнения могут развиться после системной абсорбции лекарственного препарата. Таким образом, следует учитывать противопоказания к внутрипузырной инстилляции [205, 206]. 

Рекомендуется с осторожностью проводить внутрипузырную БЦЖ-терапию пациентам для минимизации осложнений, вследствие большого количества побочных эффектов по сравнению с внутрипузырной химиотерапией [207, 208, 209, 210].  

Уровень убедительности рекомендаций – А (уровень достоверности доказательств – 2).

Комментарии: БЦЖ-терапия относительно противопоказана у иммунокомпрометированных пациентов (иммуносупрессия, ВИЧ-инфекция). Серьезные побочные эффекты встречаются менее чем у 5 \% пациентов и в большинстве случаев могут быть эффективно излечены. Показано, что поддерживающая схема лечения не ассоциирована с повышенным риском побочных эффектов в сравнении с индукционным курсом терапии. Некоторые небольшие исследования показали аналогичную эффективность и отсутствие увеличения количества осложнений по сравнению с не иммунокомпрометированными пациентами. В связи с тем, что БЦЖ-терапия слабо влияет на опухоли с низким риском развития рецидива, рекомендовано рассматривать ее как излишнее лечение для этой когорты пациентов [210]. 

Также отмечено, что у БЦЖ-терапии больше побочных эффектов, чем у ХТ. По этой причине оба вида лечения (БЦЖ-терапия и внутрипузырная ХТ противоопухолевыми антибиотиками) остаются возможными методами терапии. При окончательном его выборе следует учитывать риск рецидивирования и прогрессирования для каждого пациента в отдельности так же, как и эффективность и побочные эффекты любого метода лечения. 

В случае выявления БЦЖ-рефрактерной опухоли не рекомендовано дальнейшее консервативное лечение с применением вакцины

Альтернативой БЦЖ-терапии у отобранных больных может служить внутрипузырная химиотерапия. Остается спорным вопрос о продолжительности и частоте инстилляций химиопрепаратов. Из систематического обзора литературных данных по изучению РМП, где сравнивались различные режимы внутрипузырных инстилляций химиопрепаратов, можно сделать вывод, что идеальная продолжительность и интенсивность режимов остаются неопределенными из-за противоречивых результатов. Имеющиеся данные не подтверждают эффективность проведения лечения продолжительностью более 1 года [218]. 

Адаптация рН мочи, снижение дилюции с целью сохранения концентрации химиопрепарата снижают частоту рецидивов и являются важными условиями правильно проведенной инстилляции [216, 217]. При проведении внутрипузырной химиотерапии необходимо использовать лекарственные препараты при оптимальной рН мочи и поддерживать концентрацию препарата в течение экспозиции на фоне снижения потребления жидкости.

Схемы проведения внутрипузырной химиотерапии:

Вакцина для иммунотерапии рака мочевого пузыря**: 50–100 мг вакцины, разведенной в 50 мл физиологического раствора натрия хлорида**, вводится внутрипузырно на 2 часа с рекомендацией менять положение тела каждые полчаса. Доза 50 мг предназначена для пациентов с плохой индивидуальной пNCереносимостью терапии. Индукционный курс лечения проводится по схеме: еженедельно, в течение 6 нед. Поддерживающий курс лечения проводится по одной из схем: ежемесячно в течение 1 года или трехнедельные циклы каждые 3, 6, 12, 18, 24, 30, 36 мес.[247].

Митомицин**: 40 мг в 40 мл натрия хлорида**. Первая инстилляция – в течение 6 часов после выполнения ТУР, далее еженедельно, 6–8 инстилляций. Поддерживающий курс: ежемесячно, в течение 1 года. Экспозиция – 1–2 часа. [215].

Доксорубицин**: 30-50 мг в 25-50 мл 0,9 \% раствора натрия хлорида**. В случае развития местной токсичности (химический цистит) дозу следует растворить в 50-100 мл 0,9 \% раствора натрия хлорида**. Инстилляции можно проводить с интервалом от 1 недели до 1 месяца.

Внутрипузырная химиотерапия не проводится на протяжении более чем 1 года всем пациентам НМИ РМП вне зависимости от групп риска [219]. 

3.1.2.3. Фотодинамическая терапия
Рекомендуется фотодинамическая терапия как вариант 2 линии противоопухолевой терапии у пациентов с НМИ РМП при неэффективности предшествующего лечения [271].

Уровень убедительности рекомендаций – С (уровень достоверности доказательств – 4).

Комментарии: после внутривенного введения фотосенсибилизатора (cенсибилизирующего препарата, используемого для фотодинамической/лучевой терапии) с помощью лазера проводят обработку слизистой оболочки МП. В ряде работ сообщается об уменьшении количества рецидивов после фотодинамической терапии; в настоящее время осуществляются отработка схем и накопление материала. Дозы препаратов, сроки и режимы лечения зависят от распространенности опухоли по слизистой оболочке МП, характера фотосенсибилизатора и доз лазерного излучения.

3.1.2.4. Радикальная цистэктомия
Обоснованием радикальной цистэктомии как тактики лечения немышечно-инвазивного рака мочевого пузыря являются:

– несоответствие категории рТ1 после ТУР и последующей ЦЭ регистрируется у 27-51\% пациентов [137–140];

– худший прогноз у пациентов с прогрессией до МИ РМП, по сравнению первичным МИ РМП [141–142].

У пациентов с НМИ РМП выделяют срочную (незамедлительную) радикальную цистэктомию – сразу после установления диагноза РМП без инвазии в мышечный слой и раннюю радикальную цистэктомию – после неэффективной БЦЖ-терапии. Ретроспективно показано, что пациентам РМП с высоким риском развития рецидива лучше провести раннюю, чем отсроченную, ЦЭ при выявлении рецидива опухоли после первоначального лечения с использованием ТУР и БЦЖ-терапии, тем самым улучшая результаты выживаемости [127, 132, 143].

Необходимо учитывать влияние радикальной ЦЭ на качество жизни пациентов. Потенциальный положительный эффект от радикальной ЦЭ должен быть соизмеримым с возможными рисками и показателями заболеваемости.

Рекомендуется выполнение незамедлительной радикальной ЦЭ пациентам с НМИ РМП группы высочайшего риска для достижения ремиссии [127, 128,144].

Уровень убедительности рекомендаций – С (уровень достоверности доказательств – 3).

Комментарии: группа высочайшего риска включает пациентов со следующими характеристиками: уровень инвазии рТ1G3 с CIS; множественные, рецидивные опухоли больших размеров; pT1G3 с CIS в простатическом отделе уретры; редкие гистологические варианты опухоли с плохим прогнозом; опухоли Т1 с лимфоваскулярной инвазией. Эта группа прогностически наиболее неблагоприятная. При отказе пациента от ЦЭ показана БЦЖ-терапия с поддерживающим режимом в течение 3 лет.

При отказе или противопоказаниях к радикальной цистэктомии возможно проведение повторного курса терапии вакциной для иммунотерапии рака мочевого пузыря**. 

Рекомендуется выполнение ранней радикальной ЦЭ пациентам с БЦЖ-рефрактерными опухолями для достижения ремиссии [128].

Уровень убедительности рекомендаций – С (уровень достоверности доказательств – 3).

Комментарии: отсрочка в выполнении радикальной ЦЭ может привести к снижению показателей выживаемости. У пациентов с НМИ РМП после радикальной ЦЭ показатели 5-летней безрецидивной выживаемости превышают 80 \% [144–146].

3.1.2.5. Лечение пациентов с карциномой in situ
В случае неадекватного лечения более 50 \% пациентов с ранее выявленной CIS прогрессируют в мышечно-инвазивный (МИ) РМП [128]. Считается, что сочетание pТ1G2–3 и CIS имеет более худший прогноз по сравнению с первичной или распространенной CIS и CIS простатического отдела уретры [102, 129–131]. 

Рекомендуется проведение внутрипузырной иммунотерапии вакциной для лечения рака мочевого пузыря БЦЖ** всем пациентам с самостоятельной или сопутствующей CIS. Проведение внутрипузырной БЦЖ-терапии является важным прогностическим фактором и позволяет снизить риск прогрессирования с 66 до 20\% [132].

Уровень убедительности рекомендаций – С (уровень достоверности доказательств – 4).

Комментарии: БЦЖ-терапия должна включать индукционный курс и поддерживающий режим в течение 1–3 лет.

Рекомендуется проведение иммунотерапии пембролизумабом** 200 мг 1 раз в 3 недели или 400 мг 1 раз в 6 недель в/в капельно пациентам с CIS мочевого пузыря, резистентной к внутрипузырной БЦЖ-терапии, независимо от наличия папиллярной опухоли [304, 320, 336].

Уровень убедительности рекомендаций – С (уровень достоверности доказательств – 4).

Комментарии: пембролизумаб** изучался при БЦЖ-рефрактерной CIS мочевого пузыря в несравнительном исследовании в связи с отсутствием стандартного консервативного лечения, имеющего доказанную эффективность у подобной категории больных. В когорту А однорукавного многоцентрового исследования II фазы KEYNOTE-057 вошел 101 пациент с БЦЖ-рефрактерной CIS мочевого пузыря с или без папиллярной опухоли, имеющий противопоказания к радикальной цистэктомии или отказавшийся от нее. Гиперэкспрссия PD-L1 (≥10\% по шкале CPS) имела место в 38\% случаев. Всем пациентам проводилась монотерапия пембролизумабом** (200 мг 1 раз в 3 недели, в/в капельно) с оценкой эффекта каждые 3 месяца (цистоскопия, биопсия и цитологическое исследование мочи). Запланированная длительность лечения составляла 24 месяца. Терапию завершали преждевременно при выявлении персистирующего или рецидивного немышечно-инвазивного рака мочевого пузыря высокого риска, опухолевой прогрессии, метастазирования или при развитии непереносимой токсичности. Первичной целью исследования являлась частота объективных ответов. Частота полных ответов, зарегистрированных через 3 месяца терапии, составила 39\%. Медиана длительности полного ответа равнялась 16,2 месяца. При медиане наблюдения 36,4 месяца случаев опухолевой прогрессии в мышечно-инвазивный рак не зарегистрировано. Частота нежелательных явлений 3-4 степеней тяжести составила 13\% [320]. 




\subsection{Лечение мышечно-инвазивного рака мочевого пузыря}
\label{sec:}
3.2.1. Радикальная цистэктомия
Радикальная ЦЭ является стандартным методом лечения, локализованного МИ РМП [143, 147]. Современное состояние проблемы все чаще требует более индивидуального подхода в лечении инвазивных и распространенных форм РМП. Оценка качества жизнь, работоспособность, ожидаемая продолжительность жизни, общее состояние пациента на момент операции – все это формирует новые тенденции в терапии, такие как комбинированные варианты химиолучевого лечения и органосохраняющей операции [148, 149].

Время от момента постановки диагноза до момента проведения операции точно не установлено, однако имеются данные, что выживаемость была выше в группе пациентов, которым выполнили операцию в течение 90 дней [150–152] (УД 2).

Рекомендуется выполнение радикальной ЦЭ пациентам группы высокого риска РМП при T2–4aN0M0 для достижения ремиссии [147]

Уровень убедительности рекомендаций – С (уровень достоверности доказательств – 4).

Комментарии: Показатели смертности ниже в центрах с большим опытом выполнения радикальной ЦЭ, общая 5-летняя выживаемость после ЦЭ составляет в среднем 40-60\% [153]:

− рТ1 – 75-83\%;

− рТ2 – 63-70\%;

− рТ3a – 47-53\%;

− рТ3b – 31-33\%;

− рТ4 – 19-28\%.

Рекомендуется выполнение радикальной ЦЭ пациентам, резистентным к химиолучевому лечению, при наличии свища, пациентам с тазовой болью, а также при рецидивирующей гематурии в качестве паллиативной помощи [154–156].

Уровень убедительности рекомендаций – С (уровень достоверности доказательств – 4).

Комментарии: общее количество осложнений ЦЭ составляет 9,7-30,0\%. Частота гнойно-септических осложнений достигает 0,28-30\%. Летальность после операции – 1,2-5,1\%. Интраоперационные осложнения достигают 5,3–9,7\%. Кровотечения составляют 3–7\%. Ранения прямой кишки при наличии лучевой терапии в анамнезе – 20-27\%, без лучевой терапии – 0,5-7,0 \%.

Наиболее распространенные послеоперационные осложнения [157]:

− лимфорея – 0-3 \%;

− кишечная непроходимость – 1-5 \%;

− желудочно-кишечные кровотечения – 1,5-2 \%;

− поздние послеоперационные осложнения в виде эректильной дисфункции – в 30-85 \% случаев;

− лимфоцеле – 0,1-2,6 \%;

− грыжи передней брюшной стенки – в 1,5-5,0 \% случаев.

Наличие только одного метастатического ЛУ (N1) не препятствует выполнению ортотопической пластики, но не в случае N2-3 [158].

У мужчин объем радикальной ЦЭ включает: удаление единым блоком (en bloc) мочевого пузыря с участком висцеральной брюшины и паравезикальной клетчаткой, предстательной железой и семенными пузырьками; тазовую (подвздошно-обтураторную) лимфаденэктомию. При опухолевом поражении простатической части уретры рекомендовано выполнение уретерэктомии [159, 160]. Также у мужчин возможно проведение нервосберегающей операции с сохранением кавернозных сосудисто-нервных пучков с целью профилактики развития эректильной дисфункции [159].

Женщинам рекомендован объем радикальной ЦЭ, включающий переднюю экзентерацию таза и двустороннюю тазовую лимфаденэктомию: удаление мочевого пузыря с участком висцеральной брюшины и паравезикальной клетчаткой, удаление матки с придатками, резекцию передней стенки влагалища [160].

Рекомендуется удаление регионарных лимфатических узлов в ходе выполнения радикальной ЦЭ. Выполнение расширенной лимфаденэктомии улучшает показатели выживаемости после радикальной ЦЭ по сравнению со стандартной методикой [161–165].

Уровень убедительности рекомендаций – B (уровень достоверности доказательств – 3).

Комментарии: объем тазовой лимфодиссекции включает в себя удаление ЛУ в области наружных и внутренних подвздошных сосудов, в обтураторной ямке, а также пресакральных ЛУ. Расширенная лимфодиссекция также подразумевает удаление ЛУ в области общих подвздошных сосудов до верхней границы – бифуркации аорты. Если краниальной границей служит нижняя брыжеечная артерия, то лимфодиссекция является суперрасширенной [161–165]. Оптимальный объем лимфаденэктомии не определен, однако преимущественное число рандомизированных исследований демонстрирует целесообразность выбора в пользу расширения границ лимфодиссекции как по показателям выживаемости без рецидива и прогрессии, так и по общей выживаемости [166–172].

Не рекомендуется при выполнении радикальной ЦЭ удаление уретры, которая может служить в дальнейшем для отведения мочи [173].

Уровень убедительности рекомендаций – С (уровень достоверности доказательств – 5).

Комментарии: Целесообразно сохранение уретры при отсутствии позитивного хирургического края.

3.2.1.1. Лапароскопическая и робот-ассистированная цистэктомия
Использование лапароскопической техники достаточно давно внедрено в практику и имеет большое количество публикаций, посвященных малоинвазивной методике. Эра робот-ассистированных операций – самая молодая среди всех существующих, однако число печатных работ по этой технологии конкурирует с таковыми по лапароскопии [174- 176]. Стоит отметить, что большинство представленных данных имеет низкий уровень доказательности – 4. По-видимому, это обусловлено некорректной стратификацией пациентов [174]. Лапароскопическая и робот-ассистированная ЦЭ рекомендованы к применению у пациентов с РМП, однако до сих пор остаются в фазе изучения. Лапароскопическая и робот-ассистированная техника могут применяться для лечения пациентов как с НМИ, так и с МИ РМП.

3.2.1.2. Варианты деривации мочи
Радикальная ЦЭ включает два непрерывных этапа: удаление мочевого пузыря с лимфодиссекцией и реконструктивно-пластический компонент. Вторым непрерывным этапом и является выбор способа деривации мочи [177]. Возраст >80 лет является противопоказанием к формированию резервуара [178].

Классификация видов деривации мочи:

− наружное отведение мочи (уретерокутанеостомия, кишечная пластика с формированием «сухих» и «влажных» стом);

− создание мочевых резервуаров, обеспечивающих возможность самостоятельного контролируемого мочеиспускания: орто- и гетеротопическая пластика мочевого пузыря;

− отведение мочи в непрерывный кишечник (уретеросигмостомия, операция Mainz-pouch II).

Рекомендуется при выборе способа деривации мочи подбирать метод, обеспечивающий пациенту высокий уровень качества жизни и наименьшее количество послеоперационных осложнений [177, 178, 292].

Уровень убедительности рекомендаций – С (уровень достоверности доказательств – 4).

Комментарии: тип отведения мочи не оказывает влияния на онкологические результаты. Не рекомендуется проведение лучевой терапии до оперативного вмешательства при выборе метода лечения с отведением мочи

Уретерокутанеостомия

У пациентов пожилого возраста или имеющих выраженные сопутствующие патологии предпочтительным методом является уретерокутанеостомия. Время операции, частота осложнений, пребывание в реанимации и длительность нахождения в стационаре ниже у пациентов после выведения мочеточников на кожу [179, 180]. При наружном отведении мочи пациенту необходимы мочеприемники.

Рекомендуется выполнять уретерокутанеостомию у пациентов с генерализованным или обширным местно-распространенным процессом при проведении ЦЭ с целью быстрого восстановления и проведения последующих этапов лечения [180].

Уровень убедительности рекомендаций – С (уровень достоверности доказательств – 4).

Комментарии: существует вероятность стеноза уретерокутанеостомы ввиду малого диаметра самой стомы.

Основные осложнения после операции:

− пиелонефрит;

− хроническая почечная недостаточность;

− стеноз устьев мочеточников (при формировании уретеро-уретероанастомоза «конец-в-бок»);

− стеноз стомы;

− кожные изменения вокруг стомы (мацерация, грибковое поражение).

Гетеротопический илеокондуит

Данный вариант формирования мочевого резервуара с выведением участка подвздошной кишки и формированием кутанеостомы является наиболее изученным и часто используемым. Тем не менее частота ранних послеоперационных осложнений достигает 48 \%. Пиелонефрит как наиболее частое осложнение наблюдается в 30–50 \% случаев [181].

Рекомендуется использовать илеоцекальный угол для гетеротопической пластики при операции типа Брикера для минимизации осложнений [181].

Уровень убедительности рекомендаций – С (уровень достоверности доказательств – 4).

Комментарии: наиболее часто встречающиеся осложнения [182–184]: 

− пиелонефрит;

− кишечная непроходимость;

− стеноз мочеточниково-резервуарных анастомозов;

− стеноз стомы;

− кожные изменения вокруг стомы (мацерация, грибковое поражение).

Гетеротопический илеокондуит («сухая» стома)

Рекомендуется пациентам для создания резервуара с «сухой» стомой формирование детубулярного резервуара из участка подвздошной кишки низкого давления с формированием стомы для самокатетеризации [185–190].

Уровень убедительности рекомендаций – С (уровень достоверности доказательств – 4).

Комментарии: Хорошее удерживание мочи в дневное и ночное время отмечено многими пациентами и достигает 90\% [188]. Стеноз аппендикулярной стомы встречается в 15-23\% случаев [189]. Выбор данного варианта реконструктивной пластики является достаточно трудоемким и требует навыка и опыта хирурга [190].

Ортотопический резервуар

Формирование ортотопического резервуара предполагает его расположение в полости таза, на месте удаленного МП, и создание резервуарно-уретрального анастомоза. Этот метод позволяет пациенту в дальнейшем самостоятельно контролировать акт мочеиспускания [147,190,191].

Рекомендуется выполнение ортотопической пластики каждому пациенту при отсутствии противопоказаний и вовлечения опухолью мочеиспускательного канала для улучшения качества жизни [147,190, 191,192].

Уровень убедительности рекомендаций – С (уровень достоверности доказательств – 4).

Комментарии: женщинам также возможно выполнение ортотопической пластики при условии тщательно изученной шейки мочевого пузыря (биопсия с целью выявления опухолевых участков) [192].

Рекомендуется использовать: подвздошную кишку, илеоцекальный угол, восходящую ободочную или сигмовидную кишку при формировании ортотопических мочевых резервуаров для минимизации осложнений [193, 194].

Уровень убедительности рекомендаций – С (уровень достоверности доказательств – 4).

Комментарии: противопоказания для операции – опухолевое поражение уретры ниже семенного бугорка; выраженная хроническая почечная недостаточность.

Наиболее частые осложнения [193]:

− дневное недержание мочи (5,4-30,0\%);

− ночное недержание мочи (18,6-39,0\%);

− пиелонефрит;

− метаболические осложнения (гиперхлоремический ацидоз);

− конкрементообразование;

− стриктура резервуарно-уретрального анастомоза.

3.2.2. Органосохраняющие операции
Органосохраняющее лечение мышечно-инвазивного рака мочевого пузыря направлено на сохранение пораженного органа и, как следствие, качества жизни пациентов без ухудшения выживаемости.

Рекомендуется проведение органосохраняющего лечения отобранным пациентам, соответствующим следующим критериям:

- солитарная опухоль мочевого пузыря, вне его шейки;

- категория рТ2a–b;

- грейд G1–2 или LG;

- отсутствие гидронефроза, обусловленного опухолью;

- хорошая функция мочевого пузыря до лечения;

- нормальный показатель ПСА (исследование общей и свободной фракции крови);

- отрицательный результат мультифокальной биопсии предстательной железы (опционально);

- отсутствие в анамнезе указаний на резекцию мочевого пузыря, или чреспузырную аденомэктомию, или чреспузырное удаление конкрементов мочевого пузыря;

- отсутствие в анамнезе указаний на лучевую терапию на область малого таза;

- отсутствие протяженных стриктур мочеиспускательного канала;

- противопоказания к РЦЭ [195-198].

Уровень убедительности рекомендаций – В (уровень достоверности доказательств – 3).

Не рекомендуется использование только хирургического лечения, только ХТ или только ЛТ в качестве самостоятельных методов органосохраняющего лечения МИ РМП [195,198,199].

Уровень убедительности рекомендаций – В (уровень достоверности доказательств – 3).

Комментарии: только ТУР мочевого пузыря, только ХТ или только ЛТ существенно уступают радикальной цистэктомии с НХТ или АХТ в отношении онкологических результатов, в связи с чем не рекомендуются к использованию в широкой клинической практике [148,149].

Рекомендуется использование трехмодального лечения, включающего максимальную ТУР мочевого пузыря с последующим проведением химиолучевой терапии, для сохранения мочевого пузыря отобранным пациентам с МИ РМП, соответствующих критериям, перечисленным выше [195,198,199].

Уровень убедительности рекомендаций – В (уровень достоверности доказательств – 3).

Комментарии: наиболее эффективным методом органосохраняющего лечения, который может использоваться у тщательно отобранных больных, является трехмодальная терапия, подразумевающая выполнение максимальной ТУР мочевого пузыря с последующим проведением химио-лучевой терапии (ХЛТ). Обоснованием сочетания ТУР с ЛТ является необходимость достичь полного локального контроля над первичной опухолью и регионарными лимфатическими коллекторами. Введение в схему лечения радиосенсибилизирующих цитостатиков (cенсибилизирующих препаратов, используемых для фотодинамической/лучевой терапии) направлено на усиление эффекта облучения, а также потенциально способно элиминировать микрометастазы.

В составе трехмодального лечения описаны разные схемы ХТ, включая монотерапию цисплатином** [314], а также монотерапию гемцитабином** [315]. В рандомизированном исследовании II фазы два ежедневных сеанса облучения с комбинированной ХТ (фторурацил** и цисплатин**) и один сеанс ежедневного облучения с монотерапией гемцитабином** продемонстрировали сопоставимую 3-летнюю выживаемость без отдаленных метастазов (78\% и 84\% соответственно) при большей частоте гематологических НЯ 4 степени тяжести в группе полихимиотерапии [316].

Пятилетняя специфическая и общая выживаемость больных, подвергнутых трехмодальной терапии, колеблется от 50\% до 82\% и от 36\% до 74\%, соответственно [314, 316]. Большинство рецидивов рака мочевого пузыря не инвазирует детрузор и может быть излечено консервативно. Спасительная цистэктомия требуется примерно у 10-15\% пациентов, получавших трехмодальное лечение. Отдаленные результаты спасительных операций сопоставимы с результатами первичных радикальных цистэктомий, хотя частота осложнений у облученных пациентов выше [317].

Рандомизированных исследований, сравнивающих радикалную ЦЭ и трехмодальное лечение, не проводилось. Cистематический обзор, включивший данные более 30 000 пациентов из 57 исследований, не выявил достоверных различий выживаемости между больными, подвергнутыми радикальной ЦЭ и трехмодальной терапии. Однако при сроке наблюдения 10 лет специфическая и общая выживаемость оказались выше у пациентов с сохраненным мочевым пузырем [318].  Профиль безопасности трехмодальной терапии благоприятный. Комбинированный анализ данных пациентов, входивших в 4 исследования RTOG, показал, что при медиане наблюдения 5,4 года частота поздней гастроинтестинальной и мочевой токсичности 3 степени тяжести составляет 1,9\% и 5,7\% соответственно; нежелательных явлений 4 степени тяжести не зарегистрировано [319]. Ретроспективные данные показали преимущество качества жизни пациентов, подвергнутых трехмодальной терапии, по сравнению с больными, перенесшими РЦЭ [86].

Наиболее часто используемые режимы химиолучевой терапии в составе трехмодального лечения приведены в таблице 2 [314-316, 323].

\begin{table*}[!h]
\caption{Пример простой таблицы, содержащей описательную статистику.}
\label{tab:tab_descr_1}
\setlength{\arrayrulewidth}{1.05 pt}
\renewcommand{\arraystretch}{1.1}
\begin{tabular*}{1.0\textwidth}{@{\extracolsep{\fill}}lrr}
\hline
Параметр & Название колонки & Название колонки \\
\hline
Среднее, $\mu$ & 0.79 & 0.98 \\
\hline
\end{tabular*}
\begin{spacing}{0.5}
{\scriptsize Пояснения: Здесь даются пояснения к таблице.}
\end{spacing}
\end{table*}


\subsection{Системная противоопухолевая терапия}
\label{sec:}
Схемы химиотерапии и иммунотерапии, применяемые при инвазивном и метастатическом РМП и используемые в данном разделе:

GC

гемцитабин** – 1000 мг/м2 в/в в 1-й, 8-й и 15-й день 

цисплатин** – 70 мг/м2 в/в в 1 (2)й день + гидратация - изотонический раствор натрия хлорида** (≈ 2,5л), с целью поддержания диуреза > 100 мл/ч в процессе введения цисплатина** и в последующие 3 ч [224].

Цикл повторяют каждые 4 нед.

GemCarbo

#гемцитабин** – 1000 мг/м2 в/в в 1-й и 8-й дни

карбоплатин** – AUC-4-5 в 1-й день (дозовый режим может быть изменен в зависимости от клинической ситуации - вынужденная редукция или эскалация дозы в пределах AUC-3-6)

Цикл повторяют каждые 3 нед [241].

MVAC

#винбластин** – 3 мг/м2 в/в во 2-й, 15-й, 22-й дни 

доксорубицин** – 30 мг/м2 в/в во 2-й день 

метотрексат** – 30 мг/м2 в/в в 1-й, 15-й, 22-й дни 

цисплатин** – 70 мг/м2 во 2-й день + гидратация 

Цикл повторяют каждые 4 нед [239].

DD-MVAC

#винбластин** – 3 мг/м2 в/в во 2-й, 

доксорубицин** – 30 мг/м2 в/в во 2-й день 

метотрексат** – 30 мг/м2 в/в в 1-й, 

#цисплатин** – 70 мг/м2 во 2-й день + гидратация 

рчГ-КСФ

Цикл повторяют каждые 2 нед [293].

MCV

#винбластин** – 4 мг/м2 в/в в 1-й, 8-й дни,

метотрексат** – 30 мг/м2 в/в в 1-й, 8-й дни 

цисплатин** – 100 мг/м2 во 2-й день + гидратация 

кальция фолинат** 15 мг в/в каждые 6 ч №4 во 2-й и 9-й дни

Цикл повторяют каждые 4 нед. [339]

Винфлунин – внутривенно медленно в течение 20 минут, по 320 мг/м2 каждые 3 недели (дозовый режим может быть изменен в зависимости от клинической ситуации - редукция дозы до 250-280 мг/м2)[243, 257, 309].

#Паклитаксел** 80 мг/м2 в 1, 8, 15-й дни; цикл 28 дней, или 175  мг/м2 в 1-й день; цикл 21 день [243, 257,309, 332].

#Доцетаксел** 75–100 мг/м2 в 1-й день; цикл 21 день [243, 257, 309, 333, 334].

#Гемцитабин** 1000–1250 мг/м2 в/в в 1, 8, 15-й дни; цикл 28 дней  [326-328, 335].

Иммуноонкологические препараты (моноклональные антитела):  

Атезолизумаб** – 840 мг в виде в/в инфузии каждые 2 недели, или 1200 мг в виде в/в инфузии каждые 3 недели, или 1680 мг в виде в/в инфузии каждые 4 недели. Первую дозу атезолизумаба** необходимо вводить в течение 60 минут. При хорошей переносимости первой инфузии все последующие введения можно проводить в течение 30 минут [288].   

Пембролизумаб** – 200 мг в виде в/в инфузии в течение 30 минут каждые 3 недели, или 400 мг в виде в/в инфузии в течение 30 минут каждые 6 недель [289, 294, 336].

Авелумаб** – 800 мг в виде в/в инфузии в течение 60 минут каждые 2 недели [304]

Ниволумаб** – 3 мг/кг или 240 мг в виде в/в инфузии каждые 2 недели, либо 480 мг в виде в/в инфузии каждые 4 недели. Первое введение должно быть осуществлено в течение 60 минут, при хорошей переносимости все последующие – на протяжении 30 минут [290, 291].

Оценка эффективности химиотерапии проводится на основании критериев ответа солидных опухолей на лечение (RECIST 1.1.). Оценка эффективности иммунотерапии проводится на основании критериев ответа солидных опухолей на лечение (iRECIST 1.1.). [приложение Г3].

Последовательность самостоятельной системной терапии первой и второй линий представлены в таблицах на стр. 116-117 [приложение Б].

3.3.1. Неоадъювантная химиотерапия
Применение только хирургического лечения обеспечивает 5-летнюю выживаемость лишь у 50\% пациентов МИ РМП [191, 220, 221]. С целью улучшения этих результатов более 30 лет применяется неоадъювантная платиносодержащая химиотерапия [222]. Несмотря на столь длительный период использования этого режима терапии, увеличение выживаемости не превышает 8\% [223].

Рекомендуется проведение неоадъювантной ХТ с включением схем на основе цисплатина** пациентам со стадией сТ2-Т4aN0/+M0 при наличии сохраненной функции почек (клиренс креатинина >60 мл/мин) и общего удовлетворительного состояния (ECOG <2) для уменьшения объема опухоли, воздействия на субклинические микрометастазы, повышения резектабельности опухоли и повышения выживаемости пациентов [223-228].

Уровень убедительности рекомендаций – А (уровень достоверности доказательств – 1).

Комментарии: терапию проводят перед хирургическим или лучевым лечением. Главное преимущество неоадъювантной ХТ – возможность оценить ее воздействие на первичный очаг, что может влиять на тактику дальнейшего лечения [145].

Рекомендуется использовать схемы неоадъювантной химиотерапии: GC, MVAC, DD-MVAC или MCV для увеличения выживаемости пациентов с МИ РМП и стадией сТ2-Т4аN0/+M0 [228, 293, 339] (расшифровка рекомендуемых схем дана в начале раздела 3.2.2. Химиотерапия).

Уровень убедительности рекомендаций – С (уровень достоверности доказательств – 4).

Комментарии: при использовании цисплатин**-содержащих схем, по разным данным, эффект был достигнут у 40-70\% пациентов. По результатам рандомизированных исследований продемонстрировано статистически значимое увеличение общей выживаемости на 5–8 \% среди получавших неоадъювантную ХТ [223-228].

Не рекомендуется всем пациентам с РМП проведение неоадъювантной ХТ в монорежиме [295].

Уровень убедительности рекомендаций – C (уровень достоверности доказательств – 4).

3.3.2. Адъювантная химиотерапия
В настоящее время продолжается дискуссия о целесообразности проведения адъювантной ПХТ у пациентов с высоким риском рецидива заболевания после радикальной (R0) [229]. Некоторые авторы считают, что адъювантная ХТ позволяет улучшить отдаленные результаты лечения в данной группе пациентов в среднем на 20-30\%. Однако вопрос о целесообразности адъювантного лечения, оптимальном режиме химиотерапии и о сроках ее проведения остается предметом клинических исследований. В настоящее время адъювантная ХТ по схемам: GC, MVAC или MCV, может быть рекомендована пациентам с рТ2–4N0/+М0R0, не получивших неоадъювантной ХТ [230-232].

Не рекомендуется рутинное применение адъювантной ХТ после хирургического лечения у пациентов с МИ РМП [229].

Уровень убедительности рекомендаций – В (уровень достоверности доказательств – 2).

Рекомендуется проведение адъювантной химиотерапии пациентам с МИ РМП стадии рТ2–4N0/+М0R0, соматически сохранным, способным перенести не менее 4 курсов химиотерапии после радикальной операции для увеличения продолжительности жизни [233-235].

Уровень убедительности рекомендаций – А (уровень достоверности доказательств – 2).

Комментарии: проводились рандомизированные исследования с применением различных схем адъювантной ХТ; в большинстве из них были получены данные о продлении безрецидивного периода по сравнению с контрольной группой (только радикальная ЦЭ) [233-235].

Рекомендуется проведение адъювантной терапии #ниволумабом** в дозе 240 мг 1 раз в 2 недели внутривенно в течение 1 года пациентам с МИ РМП со стадией рТ2–4N0/+М0R0, независимо от статуса PD-L1 и проведения НХТ [302, 303].

Уровень убедительности рекомендаций – А (уровень достоверности доказательств – 2).

Комментарий: по данным рандомизированного исследования III фазы CheckMate 274 (включившем 709 радикально оперированных больных уротелиальным раком группы высокого риска прогрессирования (T2-4 и/или N+)) адъювантная иммунотерапия #ниволумабом** достоверно увеличивает безрецидивную выживаемость, выживаемость без рецидива за пределами мочевыводящих путей, а также выживаемость без прогрессирования независимо от статуса PD-L1 и проведения неоадъювантной химиотерапии. Адъювантная иммунотерапия ассоциирована с благоприятным профилем безопасности и не ухудшает качество жизни пациентов [304].

3.3.3. Системная противоопухолевая терапия при неоперабельном местно-распространенном и метастатическом РМП
Первая линия лекарственной терапии

Выбор метода лекарственной терапии осуществляется на основании наличия противопоказаний к назначению цисплатина**, противопоказаний к назначению карбоплатина** и экспрессии PD-L1 в опухолевой ткани.

Рекомендуется пациентам с неоперабельным местно-распространенным и диссеминированным РМП, не имеющим противопоказаний к назначению цисплатина**, в первой линии терапии назначать химиотерапию в режимах GC или MVAC или DD-MVAC [234,239].

Уровень убедительности рекомендаций А (уровень достоверности доказательств – 2).

Комментарии: противопоказанием к назначению цисплатина** является наличие не менее одного из следующих критериев: соматический статус по классификации Eastern Cooperatve Oncology Group (ECOG) > 1; скорость клубочковой фильтрации (СКФ) ≤ 60 мл/мин/1,73 м2; снижение слуха ≥ 2 степени; периферическая нейропатия ≥ 2 степени или сердечная недостаточность класса III по классификации Нью-Йоркской кардиологической ассоциации [240].

В рандомизированном исследовании III фазы (n 405) больные уротелиальным раком IV стадии, не получавшие предшествующей терапии, были рандомизированы на ХТ по схеме GC или M-VAC. Режимы продемонстрировали сопоставимые частоту объективного ответа, время до прогрессирования и 18-месячную общую выживаемость. Наиболее значимыми видами токсичности являлись миелотоксичность, сепсис на фоне фебрильной нейтропении и мукозит. У больных, получавших GC, чаще отмечались тяжелая анемия и тромбоцитопения; в группе, получавшей M-VAC, чаще регистрировались тяжелая, фебрильная нейтропения, а также тяжелые мукозиты [234].

Крупное рандомизированное исследование фазы III сравнивало DD-MVAC с поддерживающей терапией гранулоцитарными колониестимулирующими факторами со стандартным MVAC. DD-MVAC увеличивал частоту объективного ответа, однако не приводил к значимому увеличению медианы общей выживаемости. У пациентов, получавших DD-MVAC c гранулоцитарными колониестимулирующими факторами, наблюдалась меньшая общая токсичность [239].

Рекомендуется выполнять патолого-анатомическое исследование биопсийного (операционного) материала с применением иммуногистохимической оценки экспрессии PD-L1 всем пациентам с неоперабельным местно-распространенным и диссеминированным РМП в первой линии терапии, которые имеют противопоказания к назначению цисплатина**. При планировании терапии пембролизумабом** оценка экспрессии PD-L1 должна производиться по шкале CPS с использованием клона 22С3 [299,304], атезолизумабом** – по шкале IC с использованием клона SP 142 [244,305].

Уровень убедительности рекомендаций С (уровень достоверности доказательств – 4).

Комментарии: основанием для регистрации ингибиторов PD-(L)1 для первой линии терапии распространенного уротелиального рака у пациентов с противопоказаниями к цисплатину**, послужили исследования II фазы, в которых применялось PD-L1 тестирование опухолевой ткани. В исследовании пембролизумаба** использовалась комбинированная шкала оценки экспрессии PD-L1 (CPS), учитывающая позитивные клетки опухоли и клетки иммунной системы, инфильтрирующие опухоль [299]; в исследовании атезолизумаба** учитывалось окрашивание только иммунных клеток [244]. Результаты применения данных препаратов при оценке экспрессии по иным шкалам не изучались. В связи с этим, для селекции кандидатов для иммунотерапии пембролизумабом** и атезолизумабом** необходимо тестирование с использованием шкал с доказанной предикторной ценностью. 

Рекомендуется пациентам с неоперабельным местно-распространенным и диссеминированным РМП, имеющим противопоказания к назначению цисплатина** и гиперэкспрессию PD-L1 в опухолевой ткани, проведение иммунотерапии:

- при гиперэкспрессии PD-L1 ≥10\% - монотерапии пембролизумабом** (200 мг в виде в/в инфузии в течение 30 минут каждые 3 недели или 400 мг 1 раз в 6 недель) [299, 304, 336];

- при гиперэкспрессии PD-L1 ≥5\% – монотерапии атезолизумабом** (840 мг в виде в/в инфузии каждые 2 недели, или 1200 мг в виде в/в инфузии каждые 3 недели, или 1680 мг в виде в/в инфузии каждые 4 недели) [244, 305, 337].

Уровень убедительности рекомендаций В (уровень достоверности доказательств – 3).

Комментарии: эффективность и безопасность пембролизумаба** в первой линии терапии распространенного уротелиального рака изучались в рамках многоцентрового исследования II фазы KEYNOTE-052, включившего 374 больных, имевших противопоказания к терапии цисплатином**. Первичной целью являлась частота объективного ответа у всех пациентов и у больных с гиперэкспрессией PD-L1. Оценка PD-L1-статуса проводилась по CPS. Пограничное значение экспрессии PD-L1 было выделено у первых 100 больных и составило 10\%. Частота объективного ответа у всех больных составила 24\%, у пациентов с экспрессией PD-L1 ≥10\% - 38\%. Медиана времени до ответа равнялась 2 месяца, при медиане наблюдения 5 месяцев 83\% ответов продолжались, медиана длительности ответа не достигнута. Наиболее распространенными нежелательными явлениями 3-4 степени тяжести, связанными с лечением, являлись слабость (2\%), повышение уровня сывороточной щелочной фосфатазы (1\%) и снижение мышечной силы (1\%) [299].

Ингибитор PD-L1 атезолизумаб** в первой линии терапии распространенного уротелиального рака у больных с противопоказаниями к терапии цисплатином** изучался в 1 когорте исследования IMvigor210. Статус экспрессии PD-L1 на инфильтрирующих лимфоцитах в микроокружении опухоли определяли как процент позитивных иммунных клеток: IC0 (<1\%), IC1 (≥1\% но <5\%) и IC2/3 (≥5\%). Первичной целью являлась частота объективного ответа, которая составила 23\% у всех пациентов и достигла 28\% у больных с гиперэкспрессией PD-L1 IC2/3. При медиане наблюдения 17,2 мес медиана длительности ответа не достигнута. Нежелательные явление, связанные с лечением, наблюдались у 66\% (3-4 степени тяжести – у 16\%) больных [244].

Рекомендуется пациентам с неоперабельным местно-распространенным и диссеминированным РМП, имеющим противопоказания к назначению цисплатина** и отсутствие гиперэкспрессии PD-L1 в опухолевой ткани, проведение химиотерапии в режиме GemCarbo [241].

Уровень убедительности рекомендаций А (уровень достоверности доказательств – 2).

Комментарии: рандомизированное исследование II/III фазы EORTC 30986 сравнивало две схемы, содержащие карбоплатин** (метотрексат**, карбоплатин**, винбластин** (M-CAVI) и GemCarbo, у пациентов с такими противопоказаниями к цисплатину**, как СКФ <60 мл/мин/1,73 м2 и/или соматический статус ECOG 2. Оба режима продемонстрировали противоопухолевую активность: частота объективного ответа составила 42\% для GemCarbo и 30\% для M-CAVI. Частота тяжелых нежелательных явлений достигла 13,6\% и 23\% в группах исследования, соответственно [241]. На основании этих данных комбинация GemCarbo стала стандартом лечения этой группы пациентов.

Рекомендуется пациентам неоперабельным местно-распространенным или диссеминированным уротелиальным раком мочевого пузыря, достигшим контроля над опухолью (полный, частичный ответ или стабилизация опухолевого процесса) после 4-6 циклов химиотерапии, основанной на препаратах платины, проведение поддерживающей терапии авелумабом**. [301,302]

Уровень убедительности рекомендаций – А (уровень достоверности доказательств – 2).

Комментарии: рандомизированное клиническое исследование III фазы JAVELIN Bladder 100 изучало влияние поддерживающей терапии ингибитором PD-L1 авелумабом** после первой линии лечения комбинацией препарата платины и гемцитабина** у больных распространенным уротелиальным раком с объективным ответом или стабилизацией опухолевого процесса после 4–6 циклов химиотерапии.  Больных рандомизировали в группу авелумаба** или наилучшей поддерживающей терапии. Авелумаб** значимо увеличивал общую выживаемость с 14,3 до 21,4 месяца (HR: 0,69; 95\% СI: 0,56–0,86; p <0,001). Нежелательные явления ≥3 степени тяжести наблюдались у 47\% больных группы авелумаба** по сравнению с 25\% пациентов группы контроля. Иммуно-опосредованные нежелательные явления отмечены в 29\% случаев, достигли ≥3 степени тяжести у 7\% больных и включали колит, пневмонит, сыпь, повышение уровня печеночных ферментов, гипергликемию, миозит и гипотиреоз [301, 302].

Рекомендуется пациентам с неоперабельным местно-распространенным и диссеминированным РМП, не имеющим противопоказаний к назначению препаратов платины, в первой линии терапии назначать иммуно-химиотерапию гемцитабином** с препаратом платины и #атезолизумабом** независимо от экспрессии PD-L1 [300].

- пациентам без противопоказаний к цисплатину**:

#гемцитабин** – 1000 мг/м2 в/в в 1-й и 8-й дни

цисплатин** – 70 мг/м2 в/в в 1-й (2-й) день + гидратация - изотонический раствор натрия хлорида** (≈ 2,5 л), с целью поддержания диуреза > 100 мл/ч в процессе введения цисплатина** и в последующие 3 ч.

Цикл повторяют каждые 3нед.

#атезолизумаб** – 1200 мг в/в капельно, каждые 3 недели.

- пациентам с противопоказаниями к цисплатину**:

#гемцитабин** – 1000 мг/м2 в/в в 1-й и 8-й дни

карбоплатин** – AUC-4,5 в 1-й день

Цикл повторяют каждые 3 нед

#атезолизумаб** – 1200 мг в/в капельно, каждые 3 недели

Уровень убедительности рекомендаций А (уровень достоверности доказательств – 2).

Комментарии: рандомизированное исследование IMvigor130 сравнивало комбинацию ингибитора PD-L1 атезолизумаба** с ХТ в режимах GC/GemCarbo с ХТ GC/GemCarbo в сочетании с плацебо или монотерапией атезолизумабом**. В исследовании была достигнута первичная конечная точка: иммуно-химиотерапия обеспечивала преимущество беспрогрессивной выживаемости по сравнению с химиотерапией и плацебо во всей популяции больных (8,2 и 6,3 месяца, соответственно; HR: 0,82 (95\% CI: 0,70–0,96); p = 0,007). Незрелые данные по ОВ при медиане наблюдения 11,8 месяца не продемонстрировали различий между группами. Из-за иерархического дизайна тестирования сравнение ХТ с монотерапией атезолизумабом** еще не проводилось [300].

Рекомендуется пациентам с неоперабельным местно-распространенным и диссеминированным РМП, имеющим противопоказания к назначению карбоплатина** проведение иммунотерапии независимо от гиперэкспрессии PD-L1 в опухолевой ткани:

- монотерапии пембролизумабом** (200 мг в виде в/в инфузии в течение 30 минут каждые 3 недели или 400 мг 1 раз в 6 недель) [299, 304, 337];

- монотерапии атезолизумабом** (840 мг в виде в/в инфузии каждые 2 недели, или 1200 мг в виде в/в инфузии каждые 3 недели, или 1680 мг в виде в/в инфузии каждые 4 недели) [244, 305].

Уровень убедительности рекомендаций В (уровень достоверности доказательств – 3).

Комментарии: эффективность и безопасность #пембролизумаба** в первой линии терапии распространенного уротелиального рака изучались в рамках многоцентрового исследования II фазы KEYNOTE-052, включившего 374 больных, имевших противопоказания к терапии цисплатином**. Частота объективного ответа у всех больных составила 24\%, у пациентов с экспрессией PD-L1 ≥10\% - 38\%. Медиана времени до ответа равнялась 2 месяца, при медиане наблюдения 5 месяцев 83\% ответов продолжались, медиана длительности ответа не достигнута. Наиболее распространенными нежелательными явлениями 3-4 степени тяжести, связанными с лечением, являлись слабость (2\%), повышение уровня сывороточной щелочной фосфатазы (1\%) и снижение мышечной силы (1\%) [299].

Ингибитор PD-L1 атезолизумаб** в первой линии терапии распространенного уротелиального рака у больных с противопоказаниями к терапии цисплатином** изучался в 1 когорте исследования IMvigor210. Первичной целью являлась частота объективного ответа, которая составила 23\% у всех пациентов. При медиане наблюдения 17,2 мес медиана длительности ответа не достигнута. Нежелательные явление, связанные с лечением, наблюдались у 66\% (3-4 степени тяжести – у 16\%) больных [244].

Рекомендовано пациентам с неоперабельным местно-распространенным и диссеминированным РМП, имеющим противопоказания к назначению карбоплатина**, проведение монохимиотерапии препаратами других фармакологических групп (#доцетаксел**, #паклитаксел**, #гемцитабин**) [241, 306, 307, 324-328, 338].

Уровень убедительности рекомендаций С (уровень достоверности доказательств – 4).

Комментарии: формально противопоказанием к применению карбоплатина** является выраженное снижение функции костного мозга. Однако в клинической практике у больных распространенным уротелиальным раком в качестве факторов, исключающих возможность назначения карбоплатина**, используются критерии, заимствованные из рандомизированного исследования EORTC 30986 (низкий соматический статус ECOG >2, СКФ <30 мл/мин/1,73 м2 или комбинация соматического статуса ECOG 2 и СКФ <60 мл/мин/1,73 м2), так как прогноз этой популяции пациентов плохой независимо от проведения ХТ на основе препаратов платины или без них [241]. Данные о возможностях лекарственного противоопухолевого лечения у данной группы пациентов ограничены отдельными однорукавными исследованиями, показавшими приемлемую эффективность и безопасность монотерапии таксанами [306, 307] и гемцитабином** [324-328]. Имеющейся доказательной базы недостаточно для формирования клинических рекомендаций.

Вторая линия лекарственной терапии

Рекомендуется в качестве режима предпочтения назначение монотерапии пембролизумабом** больным неоперабельным местно-распространенным или метастатическим РМП с прогрессированием после или на фоне проведения химиотерапии, основанной на препаратах платины [257].

Уровень убедительности рекомендаций А (уровень достоверности доказательств – 2).

Комментарии: рандомизированное исследование III фазы KEYNOTE-045 было направлено на сравнение эффективности пембролизумаба** и традиционной химиотерапии у больных неоперабельным местно-распространенным и диссеминированным уротелиальным раком, прогрессирующим на фоне или в течение 12 месяцев после завершения ХТ, основанной на цисплатине**. В исследование было включено 542 пациента, рандомизированного на терапию пембролизумабом** или монохимиотерапию (паклитаксел**, доцетаксел** или винфлунин). Первичной целью являлась оценка общей и беспрогрессивной выживаемости во всей популяции исследования и у больных с экспрессией PD-L1 ≥10\% по CPS. При медиане наблюдения 18,5 месяца пембролизумаб** значимо увеличивал медиану общей выживаемости с 7,4 до 10,3 месяца (HR 0,70; 95\% CI:0,57-0,86; p = 0,0004). Различия беспрогрессивной выживаемости между группами были недостоверны (медиана – 2,1 в группе пембролизумаба** vs 3,3 месяца в группе химиотерапии, 18-месячная – 16,8\% vs 3,5\% соответственно; р=0,32). Частота объективного ответа и полного ответа в группе пембролизумаба** составила 21,1\% и 7,8\%, в группе химиотерапии – 11,0\% и 2,9\% соответственно. Медиана длительности ответа на фоне терапии пембролизумабом** не достигнута, на фоне химиотерапии – 4,4 месяца. Наличие экспрессии PD-L1 (CPS≥10\%) не оказывало влияния на частоту объективного ответа и показатели выживаемости. Иммунотерапия лучше переносилась пациентами: любые нежелательные явления, связанные с лечением, зарегистрированы у 61,3\% больных в группе пембролизумаба** и у 90,2\% пациентов, получавших химиотерапию; токсичность ≥ 3 степени тяжести зарегистрирована у 16,5\% и 49,8\% больных соответственно [257].

Рекомендуется в качестве альтернативного режима назначение монотерапии ниволумабом** больным неоперабельным местно-распространенным или метастатическим РМП с прогрессированием после или на фоне проведения химиотерапии, основанной на препаратах платины [246,308].

Уровень убедительности рекомендаций С (уровень достоверности доказательств – 4).

Комментарии: ниволумаб** был изучен в качестве монотерапии при диссеминированном уротелиальном раке в исследовании I/II фазы CheckMate 032 у пациентов, получавших не менее 1 предшествующей линии лечения, включавшего препараты платины, независимо от статуса PD-L1. Первичной целью исследования являлась частота объективного ответа, которая составила 24,4\% и не зависела от уровня экспрессии PD-L1. Связанные с лечением нежелательные явления 3-4 степени тяжести развились у 22\% пациентов; наиболее частыми из них были повышение сывороточной липазы (5\%) и амилазы (4\%) [308].

В исследовании II фазы Checkmate 275 (n 270) ниволумаб** во второй линии терапии резистентного к препаратам платины уротелиального рака позволил добиться объективного ответа в 19,6\% случаев при медиане времени до лечебного эффекта 1,9 мес. Частота объективного ответа нарастала по мере увеличения уровня экспрессии PD-L1 и составила 28,4\% при положительном окрашивании ≥5\% клеток, и 16,1\% у пациентов с экспрессией PD-L1 <5\%. Медиана беспрогрессивной выживаемости составила 2 месяца (1,87 месяца - при экспрессии PD-L1 <1\% и 3,6 месяца - ≥1\%). Медиана общей выживаемости равнялась 8,74 месяца у всех больных (5,95 месяца – при экспрессии PD-L1 <1\% и 11,3 месяца - при экспрессии PD-L1 ≥1\%). НЯ 3-4 степени тяжести, связанные с лечением, имели место в 18\% наблюдений [246].

Рекомендуется в качестве альтернативного режима назначение монотерапии атезолизумабом** больным неоперабельным местно-распространенным или метастатическим РМП с прогрессированием после или на фоне проведения химиотерапии, основанной на препаратах платины или препаратах других фармакологических групп [301,310,311].

Уровень убедительности рекомендаций С (уровень достоверности доказательств – 4).

Комментарии: терапия атезолизумабом** при резистентных опухолях изучалась во 2 когорте исследования IMvigor 210, включившей 315 больных распространенным уротелиальным раком, ранее получавших препараты платины. Первичной целью исследования являлась оценка частота объективного ответа, которая составила 16\% у всех больных и достигла 28\% при PD-L1 IC2/3 [310].

IMvigor211 – рандомизированное исследование III фазы, сравнивавшее эффективность и безопасность атезолизумаба** и ХТ (винфлунин, паклитаксел** или доцетаксел**) у пациентов диссеминированным уротелиальным раком, в течение или после как минимум одного цитотоксического режима терапии, основанного на препаратах платины (n 931). Исследование было отрицательным: достоверных различий общей выживаемости всей популяции пациентов, получавших атезолизумаб** или химиотерапию, не выявлено (медиана - 8,6 vs. 8,0 месяца соответственно, HR 0,85; 95\% CI: 0,73 – 0,99) [301,309].

В исследовании IIIb фазы SAUL эффективность и безопасность атезолизумаба** изучались у 1004 пациентов резистентным местно-распространенным или метастатическим уротелиальным или неуротелиальным раком мочевыводящих путей, включая больных, не соответствующих рутинным критериям включения в клинические исследования, в том числе - пациентов, получавших химиотерапию, основанную не на препаратах платины. Медиана общей выживаемости составила 8,7 месяца, медиана беспрогрессивной выживаемости - 2,2 месяца, частота объективного ответа - 13\%. Нежелательные явления ≥3 степени зарегистрированы у 45\% пациентов, что привело к прекращению лечения из-за токсичности в 8\% случаев [311].

Рекомендуется в качестве альтернативного режима назначение монотерапии винфлунином больным неоперабельным местно-распространенным или метастатическим РМП с прогрессированием после или на фоне проведения химиотерапии, основанной на препаратах платины [242].

Уровень убедительности рекомендаций С (уровень достоверности доказательств – 4).

Комментарии: в рандомизированное исследование III фазы, сравнивавшее винфлунин с наилучшей поддерживающей терапией при распространенном уротелиальном раке у больных с прогрессированием после проведения химиотерапии, основанной на цисплатине**, вошло 370 больных. Винфлунин продемонстрировал недостоверное преимущество общей выживаемости во всей популяции пациентов по сравнению с поддерживающим лечением (6,9 vs. 4,6 месяца соответственно, HR 0,88; 95\% CI, 0,69-1,12: P = 0,287). Однако при анализе фактических лечебных групп разница результатов в пользу винфлунина оказалась статистически значимой в отношении общей выживаемости (6,9 vs. 4,3 соответственно, P = 0,04), а также частоты объективного ответа (16\% vs 0\%, P = 0,0063), контроля над болезнью (41,1\% vs 24,8\%, P = 0,0024) и медианы БПВ (3,0 vs 1,5 месяца, P = 0,0012). Длительность объективного ответа на терапию винфлунином составила 7,4 месяца (95\% CI 4,5 – 17,0 месяца) [242].

Рекомендуется в качестве альтернативного режима назначение монотерапии #гемцитабином** или #паклитакселом** или #доцетакселом** больным неоперабельным местно-распространенным или метастатическим РМП с прогрессированием после или на фоне проведения химиотерапии, основанной на препаратах платины [243, 257, 309, 338].

Уровень убедительности рекомендаций С (уровень достоверности доказательств – 4).

Комментарии: IMvigor211 – рандомизированное исследование III фазы, сравнивавшее эффективность и безопасность атезолизумаба** и ХТ (винфлунин, паклитаксел** или доцетаксел**) у пациентов диссеминированным уротелиальным раком, в течение или после как минимум одного цитотоксического режима терапии, основанного на препаратах платины (n 931). Монохимиотерапия не уступала иммунотерапии в отношении общей выживаемости у всей популяции пациентов (медиана - 8,6 vs. 8,0 месяца соответственно, HR 0,85; 95\% CI: 0,73 – 0,99) [301,309].

Рекомендуется в качестве альтернативного режима назначение монотерапии #гемцитабином** больным неоперабельным местно-распространенным или метастатическим РМП с прогрессированием после или на фоне проведения химиотерапии, основанной на препаратах платины [325-328].

Уровень убедительности рекомендаций С (уровень достоверности доказательств – 4).

Комментарии: активность #гемцитабина** при распространенном РМП была продемонстрирован в ходе испытания I фазы, в котором четыре (27\%) из 15 пациентов, ранее получавших M-VAC, ответили на лечение. Дозы гемцитабина варьировали от 875 мг/м2 до 1307 мг/м2 в 1, 8 и 15 дни 28-дневного цикла [325]. Три последующие исследования фазы II с использованием #гемцитабина** в дозе 1200 мг/м2 в 1, 8 и 15 дни 4-недельного цикла продемонстрировали частоту ответов в диапазоне от 23\% до 28\% [326-328].




\subsection{Лучевая терапия}
\label{sec:}
% \subsection{ Лучевая терапия}
% \label{sec:}
Воздействию лучевой терапии подлежат переходно-клеточные и плоскоклеточные опухоли.  Не показано проведение ЛТ при НМИ РМП. Лучевую терапию по радикальной программе применяют при тотальном поражении стенок мочевого пузыря. При НМИ РМП дистанционную ЛТ применяют с органосохраняющей целью при быстро рецидивирующих или обширных опухолях, при которых невозможна ТУР; при высоком риске прогрессии. Описаны положительные результаты применения ЛТ у пациентов с неудачами БЦЖ-терапии. В целом ЛТ при НМИ РМП применяют редко, рандомизированных сравнительных исследований с другими методами лечения нет.

3.4.1. Самостоятельная лучевая терапия
Рекомендуется химиолучевая терапия (предпочтительно)/самостоятельная ЛТ пациентам с МИ РМП c тяжелым соматическим статусом (ECOG≥ 2, Приложение Г1), которым не показано проведение радикальной ЦЭ [259, 260]

Уровень убедительности рекомендаций – С (уровень достоверности доказательств – 4).

Комментарии: лучевой терапии могут быть подвергнуты пациенты с нормальной функцией мочевого пузыря и достаточной его емкостью при отсутствии ИМТ (режим дозирования указан ниже по тексту) [260].

Рекомендуется пациентам с небольшими (менее 5 см) солитарными опухолями МП проведение брахитерапии для достижения ремиссии [261-263].

Уровень убедительности рекомендаций – С (уровень достоверности доказательств – 4).

Комментарии: несмотря на рекомендацию, в большинстве случаев проводят дистанционную ЛТ. 

Не рекомендуется использовать у пациентов с РМП подведенную суммарную очаговую дозу при ЛТ менее 60 Гр в связи с ее малой эффективностью [264].

Уровень убедительности рекомендаций – С (уровень достоверности доказательств – 4).

Комментарии: лучевая терапия по радикальной программе проводится в режиме фракционирования с разовой очаговой дозой (РОД) 2 Гр, 5 раз в неделю до суммарной очаговой дозы (СОД) 60-66 Гр непрерывным курсом. При этом, как правило, вначале в объем облучения включается весь таз (мочевой пузырь и зоны регионарного метастазирования) до СОД 44-46 Гр, затем МП и паравезикальная клетчатка 14-16 Гр (до СОД 60 Гр), затем – локально опухоль МП 6 Гр (до СОД 66 Гр). При Т2N0M0 в совокупности с G1-2 возможно проведение радиотерапии без включения в объем облучения на 1 этапе регионарных ЛУ. При наличии протонного комплекса целесообразно использовать энергию протонного пучка 70-250 МэВ. По данным разных авторов, 5-летняя выживаемость колеблется в пределах 24-46\%. При стадии Т2 5-летняя выживаемость составляет 25,3-59,0\%, при стадии Т3 – 9-38\% и при стадии Т4 – 0-16 \%. Ответ на проведенное лечение наблюдается у 35-70\% пациентов. Частота развития местных рецидивов составляет около 50\%. Осложнения возникают у 15\% пациентов; наиболее распространенные – цистит, гематурия, дизурические явления, проктит, диарея. Более чем у 2/3 мужчин развивается эректильная дисфункция.

Возможно проведение эскалации дозы на метастатически пораженные регионарные лимфатические узлы таза (при условии соблюдения границ толерантности со стороны здоровых тканей и органов).

При наличии технических возможностей, компетенции специалистов и клинического опыта возможно рассматривать вопрос проведения режима умеренного гипофракционирования до суммарной очаговой дозы 55 Гр за 20 сеансов [321].

3.4.2. Предоперационная лучевая терапия
Рекомендуется у пациентов с МИ РМП при проведении предоперационной ЛТ суммарная очаговая доза в пределах 20-45 Гр для снижения степени инвазии опухоли и предотвращения развития местного рецидива после хирургического вмешательства [265-267]

Уровень убедительности рекомендаций – С (уровень достоверности доказательств – 4).

Комментарии: в ряде проведенных исследований показано снижение числа местных рецидивов после предоперационной ЛТ, однако в других исследованиях не отмечено ее влияния на выживаемость и частоту местного рецидивирования.

3.4.3. Послеоперационная лучевая терапия
Рекомендуется проведение послеоперационной ЛТ у пациентов с МИ РМП при местно-распространенной стадии (рТ3–4) или R+ для профилактики рецидивирования [268-270].

Уровень убедительности рекомендаций – А (уровень достоверности доказательств – 2).

Комментарии: дистанционная радиотерапия проводится на область ложа удаленной опухоли в РОД 2 Гр, 5 раз в неделю до СОД 50 Гр, затем локально на остаточную опухоль РОД 2 Гр, 5 раз в неделю, СОД 10-16 Гр (СОД за оба этапа составит 60-66 Гр). При наличии метастатического поражения регионарных ЛУ на первом этапе ЛТ в объем облучения включаются регионарные лимфатические узлы мочевого пузыря, РОД 2 Гр, 5 раз в неделю, СОД 50 Гр, затем локально, определяемые по данным КТ метастатические лимфатические узлы РОД 2 Гр, 5 раз в неделю, СОД 16 Гр (СОД за оба этапа составит 66 Гр). В связи с изменением топографо-анатомических соотношений после удаления МП отмечают увеличение постлучевых осложнений, особенно со стороны желудочно-кишечного тракта.

3.4.4. Паллиативная лучевая терапия
С целью улучшения качества жизни рекомендуется проведение паллиативной лучевой терапии пациентам с РМП для купирования или уменьшения интенсивности симптомов первичной опухоли и/или метастазов. Режим фракционирования (включая режим умеренного гипофракционирования) определяется конкретной клинической ситуацией [322, 329-331].

Уровень убедительности рекомендаций – С (уровень достоверности доказательств – 2).

Комментарии: в рандомизированном клиническом исследовании, а также в нескольких сериях наблюдений было продемонстрировано симптоматическое улучшение после провденеия паллиативной лучевой терапии у пациентов с симптомами РМП [329-331]. В многоцентровом рандомизированном исследовании ВА09 (n 500), сравнивавшем эффективность и безопасность двух режимов паллиативной лучевой терапии (35Гр в 10 фракциях и 21 Гр в 3 фракциях), проводившейся для достижения симптоматического улучшения у пациентов с противопоказаниями к другим видам лечения РМП, не было выявлено значимых различий частоты снижения интенсивности проявлений заболевания в лечебных группах (71\% - в группе 35Гр и 64\% - в группе 21Гр) и доли пациентов с зарегистрированными нежелательными явлениями; общая выживаемость в группах исследования также не различалась (HR 0,99, 95\% CI 0,82-1,21, p 0,933) [331]. Основываясь на имеющихся данных, можно рекомендовать  применение нескольких режимов облучения с РОД 3-8 Гр до СОД 8-35 Гр. Допустима реализация режимов экстремального гипофракционирования, применяющихся при IG-IMRT (SBRT) и предусматривающих подведение РОД ≥7 Гр за несколько фракций, для достижения лучшего анальгезирующего эффекта [322]. Данный режим радиотерапии возможен только в специализированных центрах, обладающих соответствующим уровнем технического оснащения, подготовленным персоналом и клиническим опытом выполнения данной технологии.



\subsection{Обезболивание}
\label{sec:}
Принципы обезболивания и оптимального выбора противоболевой терапии у пациентов с РМП при наличии хронического болевого синдрома соответствуют принципам обезболивания, изложенным в методических рекомендациях «Практические рекомендации по лечению хронического болевого синдрома у онкологических больных» (Коллектив авторов: Когония Л.М., Волошин А.Г., Новиков Г.А., Сидоров А.В., DOI:10.18 027 / 2224–5057–2018–8–3s2–617–635,https://rosoncoweb.ru/standarts/RUSSCO/2018/2018-47.pdf).

\subsection{Сопроводительная терапия у пациентов с РМП}
\label{sec:}
% \subsection{Сопроводительная терапия у пациентов с РМП}
% \label{sec:}
Принципы профилактики и лечения анемии у пациентов с РМП соответствуют принципам, изложенным в клинических рекомендациях «Анемия при злокачественных новообразованиях», размещенным в рубрикаторе клинических рекомендаций Минздрава России https://cr.minzdrav.gov.ru.

Принципы лечения и профилактики тошноты и рвоты у пациентов с РМП соответствуют принципам, изложенным в методических рекомендациях «Профилактика и лечение тошноты и рвоты» (Коллектив авторов: Владимирова Л.Ю., Гладков О.А., Когония Л.М., Королева И.А., Семиглазова Т.Ю., DOI: 10.18 027/2224–5057–2018–8–3s2–502–511, https://rosoncoweb.ru/standarts/RUSSCO/2018/2018-35.pdf).

Принципы лечения и профилактики костных осложнений у пациентов с РМП соответствуют принципам, изложенным в методических рекомендациях «Использование остеомодифицирующих агентов для профилактики и лечения патологии костной ткани при злокачественных новообразованиях» (Коллектив авторов: Манзюк Л.В., Багрова С.Г., Копп М.В., Кутукова С.И., Семиглазова Т.Ю., DOI: 10.18 027/2224–5057–2018–8–3s2–512–520, https://rosoncoweb.ru/standarts/RUSSCO/2018/2018-36.pdf).

Принципы профилактики и лечения инфекционных осложнений и фебрильной нейтропении у пациентов с РМП соответствуют принципам, изложенным в методических рекомендациях «Лечение инфекционных осложнений фебрильной нейтропении и назначение колониестимулирующих факторов» (Коллектив авторов: Сакаева Д.Д., Орлова Р.В., Шабаева М.М., DOI: 10.18 027 / 2224–5057–2018–8–3s2–521–530, https://rosoncoweb.ru/standarts/RUSSCO/2018/2018-37.pdf).

Принципы профилактики и лечения гепатотоксичности у пациентов с РМП соответствуют принципам, изложенным в методических рекомендациях «Коррекция гепатотоксичности» (Коллектив авторов: Ткаченко П.Е., Ивашкин В.Т., Маевская М.В., DOI: 10.18 027/2224–5057–2018–8–3s2–531–544, https://rosoncoweb.ru/standarts/RUSSCO/

2018/2018-38.pdf).

Принципы профилактики и лечения сердечно-сосудистых осложнений у пациентов с РМП соответствуют принципам, изложенным в методических рекомендациях «Практические рекомендации по коррекции кардиоваскулярной токсичности противоопухолевой лекарственной терапии» (Коллектив авторов: Виценя М.В., Агеев Ф.Т., Гиляров М.Ю., Овчинников А.Г., Орлова Р.В., Полтавская М.Г., Сычева Е.А., DOI: 10.18 027/2224–5057–2018–8–3s2–545–563, https://rosoncoweb.ru/standarts/RUSSCO/

2018/2018-39.pdf).

Принципы профилактики и лечения кожных осложнений у пациентов с РМП соответствуют принципам, изложенным в методических рекомендациях «Практические рекомендации по лекарственному лечению дерматологических реакций у пациентов, получающих противоопухолевую лекарственную терапию» (Коллектив авторов: Королева И.А., Болотина Л.В., Гладков О.А., Горбунова В.А., Круглова Л.С., Манзюк Л.В., Орлова Р.В., DOI: 10.18 027/2224–5057–2018–8–3s2–564–574, https://rosoncoweb.ru/standarts/RUSSCO/2018/2018-40.pdf).

Принципы нутритивной поддержки у пациентов с РМП соответствуют принципам, изложенным в методических рекомендациях «Практические рекомендации по нутритивной поддержке онкологических больных» (Коллектив авторов: Сытов А.В., Лейдерман И.Н., Ломидзе С.В., Нехаев И.В., Хотеев А.Ж., DOI: 10.18 027/2224–5057–2018–8–3s2–575–583, https://rosoncoweb.ru/standarts/RUSSCO/2018/2018-41.pdf).

Принципы профилактики и лечения нефротоксичности у пациентов с РМП соответствуют принципам, изложенным в методических рекомендациях «Практические рекомендации по коррекции нефротоксичности противоопухолевых препаратов» (Коллектив авторов: Громова Е.Г., Бирюкова Л.С., Джумабаева Б.Т., Курмуков И.А., DOI: 10.18 027/2224–5057–2018–8–3s2–591–603, https://rosoncoweb.ru/standarts/RUSSCO/2018/

2018-44.pdf).

Принципы профилактики и лечения тромбоэмболических осложнений у пациентов с РМП соответствуют принципам, изложенным в методических рекомендациях «Практические рекомендации по профилактике и лечению тромбоэмболических осложнений у онкологических больных» (Коллектив авторов: Сомонова О.В., Антух Э.А., Елизарова А.Л., Матвеева И.И., Сельчук В.Ю., Черкасов В.А., DOI: 10.18 027/2224–5057–2018–8–3s2–604–609, https://rosoncoweb.ru/standarts/RUSSCO/2018/2018-45.pdf).

Принципы профилактики и лечения последствий экстравазации лекарственных препаратов у пациентов с РМП соответствуют принципам, изложенным в методических рекомендациях «Рекомендации по лечению последствий экстравазациипротивоопухолевых препаратов» (Автор: Буйденок Ю.В., DOI: 10.18 027/2224–5057–2018–8–3s2–610–616, https://rosoncoweb.ru/standarts/RUSSCO/2018/2018-46.pdf).

Принципы профилактики и лечения иммуноопосредованных нежелательных явлений у пациентов с РМП соответствуют принципам, изложенным в методических рекомендациях «Практические рекомендации по управлению иммуноопосредованными нежелательными явлениями» (Коллектив авторов: Проценко С.А., Антимоник Н.Ю., Берштейн Л.М., Новик А.В., Носов Д.А., Петенко Н.Н., Семенова А.И., Чубенко В.А., Юдин Д.И., DOI: 10.18 027/2224–5057–2018–8–3s2–636–665, https://rosoncoweb.ru/

standarts/RUSSCO/2018/2018-48.pdf).


\subsection{Диетотерапия}
\label{sec:}
Не рекомендуются какие-либо изменения в привычном рационе пациентов, если только они не продиктованы необходимостью коррекции коморбидных состояний или купирования/профилактики осложнений проводимого лечения (хирургического, лекарственного или лучевого) [272].

Уровень убедительности рекомендаций – С (уровень достоверности доказательств – 5).

Комментарии: Предположение о том, что сахарин является канцерогеном и вызывает РМП, основанное на экспериментальных результатах, не было подтверждено эпидемиологическими исследованиями. По данным американских исследователей, заболеваемость РМП в 1,5–2 раза выше в населенных пунктах, жители которых в течение длительного времени (40–60 лет) употребляли хлорированную воду из поверхностных источников. Роль питания в этиологии РМП остается неясной, несмотря на достаточно большое количество эпидемиологических исследований, посвященных этой проблеме.
