3.1.1.Трансуретральная резекция
Рекомендуется начинать лечение немышечно-инвазивного рака мочевого пузыря (НМИ РМП) с ТУР мочевого пузыря (за исключением пациентов с тотальным поражением МП - таким пациентам показана ЦЭ). [104, 117].

Уровень убедительности рекомендаций – С (уровень достоверности доказательств – 5).

Комментарии: При ТУР МП удаляют все видимые опухоли. Отдельно удаляют экзофитный компонент и основание опухоли. Это необходимо для правильного установления стадии заболевания (рТ), так как в зависимости от полученных результатов вырабатывают дальнейшую тактику лечения пациента. Проведение ТУР МП с последующим патоморфологическим исследованием – главный этап в лечении НМИ РМП. Целью лечения в данном случае является удаление существующей опухоли с профилактикой рецидива заболевания и предотвращением развития инфильтративной опухоли.

Наиболее распространенными осложнениями ТУР МП являются:

➢ кровотечения (интраоперационные и послеоперационные), иногда требующие открытого хирургического вмешательства;

➢ перфорация стенки мочевого пузыря (внутрибрюшинная перфорация требует лапаротомии или лапароскопии, дренирования брюшной полости, ушивания дефекта стенки мочевого пузыря).

Рекомендуется выполнять повторную ТУР (second-look) для верификации диагноза в следующих случаях:

− после неполной первоначальной ТУР – для исключения опухолей TaG1 и первичного РМП, если после первоначальной резекции в образце не было мышечной ткани;

− во всех случаях опухолей Т1;

− при всех опухолях G3 [118–122].

Уровень убедительности рекомендаций – С (уровень достоверности доказательств – 4).

Комментарии: Проведение повторной ТУР является обязательной манипуляцией у пациентов группы высокого риска. Исследования демонстрируют достоверные различия в безрецидивной выживаемости и выживаемости без прогрессии [118–121]. Повторная ТУР выполняется через 2–6 недель после первичной процедуры [122].

При некоторых экзофитных опухолях возможна резекция единым блоком (en bloc) с использованием моно- или биполярного тока, а также современных методов: лазеров (тулиевый и гольмиевый) Такая методика обеспечивает высокое качество морфологического материала с наличием мышечного слоя в 96–100\% случаев [123–126].

3.1.2. Тактика ведения пациентов с немышечно-инвазивным раком мочевого пузыря после трансуретральной резекции
3.1.2.1. Однократная немедленная внутрипузырная инстилляция химиопрепарата
При использовании ТУР можно полностью удалить макроопухоль, но невозможно повлиять на микроочаги. В результате возникают рецидивы, которые могут в дальнейшем прогрессировать до МИ РМП [117]. Поэтому необходимо рассмотреть вопрос об адъювантной терапии у всех пациентов [211]. 

Рекомендуется однократная немедленная (в первые 6 часов после ТУР) внутрипузырная инстилляция химиопрепарата (противоопухолевого антибиотика или родственного соединения) всем пациентам с НМИ РМП вне зависимости от группы риска для снижения частоты развития рецидивов [211, 212].

Уровень убедительности рекомендаций – А (уровень достоверности доказательств – 1).

Комментарии: При лечении пациентов с НМИ РМП с высокой вероятностью развития рецидива в первые 3 мес. наблюдения рекомендуется рассматривать назначение адъювантной терапии. Применение внутрипузырной химиотерапии приводит к снижению рецидивов, увеличению продолжительности безрецидивного течения, однако не сказывается на частоте прогрессирования процесса и показателях выживаемости [212]. 

Ранняя послеоперационная инстилляция не проводится в случаях явной или предполагаемой перфорации стенки мочевого пузыря, а также при гематурии, когда требуется промывание полости МП. В данном случае среднему медицинскому персоналу необходимо давать четкие инструкции по контролю свободного оттока жидкости по мочевому катетеру. Необходимость в проведении адъювантной внутрипузырной терапии зависит от прогноза рецидива заболевания[213].  

В группе пациентов низкого риска немедленная однократная химиотерапия проводится в качестве полной (завершенной) адъювантной терапии. Данной категории пациентов не требуется лечения до последующего рецидива [214]. Однако для других групп риска однократная немедленная инстилляция является недостаточной из-за высокой вероятности развития рецидива и/или прогрессирования. 

Рекомендуется проводить внутрипузырную экспозицию митомицином** или доксорубицином** всем пациентам с НМИ РМП в течение 1 часа для минимизации побочных эффектов [254, 306-308].

Уровень убедительности рекомендаций – В (уровень достоверности доказательств – 2).

Комментарии: Длительность экспозиции химиопрепарата также регламентирована. При сравнении 0,5 и 1-часовой экспозиций достоверной разницы в безрецидивной выживаемости не отмечено [254].

3.1.2.2. Адъювантная внутрипузырная терапия
Выбор тактики дальнейшего лечения и наблюдения определяетя на основании таблиц и номограмм, предложенных Европейской ассоциацией по изучению и лечению рака в 2006г. [127]. В зависимости от прогностических факторов возникновения рецидива и прогрессии у пациентов с НМИ РМП рекомендована выработка дальнейшей тактики лечения [127].

Рекомендуется проведение цистоскопии пациентам с НМИ РМП группы низкого риска после выполнения ТУР и однократной инстилляции химиопрепарата из группы противоопухолевых антибиотиков (доксорубицин**, митомицин**) с целью динамического наблюдения [127, 281].

Уровень убедительности рекомендаций – С (уровень достоверности доказательств – 2).

Комментарии: группа низкого риска – уровень инвазии рТа, дифференцировка G1, единичная опухоль менее 3 см, отсутствие CIS. Риск рецидива и прогрессирования опухоли в данной группе за 5 лет – до 37 и 1,7 \% соответственно. Смертность за 10 лет – 4,3 \%.

Рекомендуется проведение адъювантной внутрипузырной терапии вакциной для иммунотерапии рака мочевого пузыря** или химиотерапии противоопухолевым антибиотиком (доксорубицин**, митомицин**) пациентам с НМИ РМП группы промежуточного риска после выполнения ТУР и однократной инстилляции противоопухолевого антибиотика (доксорубицин**, митомицин**) с целью снижения риска рецидивов [127, 205].

Уровень убедительности рекомендаций – B (уровень достоверности доказательств – 2).

Комментарии: к этой группе относятся все пациенты, не вошедшие в группу низкого или высокого риска. Риск рецидива и прогрессирования опухоли за 5 лет – до 65 и 8 \% соответственно. Смертность за 10 лет – 12,8 \%.

Рекомендуется назначение адъювантной терапии всем пациентам с НМИ РМП группы высокого риска. Предпочтение стоит отдавать БЦЖ-терапии с поддерживающим режимом [127, 204, 205].

Уровень убедительности рекомендаций – В (уровень достоверности доказательств – 2).

Комментарии: группа высокого риска – уровень инвазии рТ1, дифференцировка G3, множественные и рецидивные опухоли; CIS, а также большие опухоли (более 3 см), pTaG1–2 при возникновении рецидива в течение 6 мес. после операции. Эта группа прогностически неблагоприятная. Эффективность внутрипузырной химиотерапии значительно ниже. Вариант выбора у данных пациентов при неэффективности комбинированного органосохраняющего лечения – ЦЭ. Риск рецидива и прогрессирования опухоли за 5 лет – до 84 и 55 \% соответственно. Смертность за 10 лет – 36,1 \%. Индукционные инстилляции вакцина для иммунотерапии рака мочевого пузыря** классически выполняются в соответствии с эмпирической 6-недельной схемой, которая была предложена Morales и соавт. [200].

Рекомендуется проведение внутрипузырной БЦЖ-терапии с использованием полной дозы в течение 1–3 лет пациентам с НМИ РМП групп промежуточного и высокого риска развития рецидива и прогрессирования для достижения ремиссии [200-204].

Уровень убедительности рекомендаций – А (уровень достоверности доказательств – 2).

Комментарии: в мета-анализе положительный эффект наблюдался только у пациентов, получивших БЦЖ-терапию по поддерживающей схеме. Используется много различных поддерживающих режимов: от 10 инстилляций, проведенных в течение 18 недель, до 27 более чем за 3 года. С помощью мета-анализа невозможно было определить, какая поддерживающая схема вакцины была наиболее эффективной. Преимущество иммунотерапии перед митомицином** в предупреждении развития рецидива и прогрессирования появляется только при применении БЦЖ-терапии продолжительностью не менее 1 года. Оптимальное количество, частота и длительность поддерживающих индукционных инстилляций остаются неизвестными. Однако результаты рандомизированного контролируемого исследования, куда вошли 1355 пациентов, показали, что проведение поддерживающей БЦЖ-терапии в течение 3 лет с использованием полной дозы вакцины снижает частоту рецидивирования по сравнению с 1 годом лечения в группе высокого риска, но это не относится к пациентам с промежуточным риском. Не наблюдалось различий при сравнении показателей прогрессирования или общей выживаемости [200–204].  

Рекомендуется пациентам с опухолью в простатической части уретры выполнение ТУР предстательной железы с последующими внутрипузырными инстилляциями вакциной для иммунотерапии рака мочевого пузыря** с целью снижения частоты рецидивов [84, 202].

Уровень убедительности рекомендаций – С (уровень достоверности доказательств – 5).

Комментарии: первые инстилляции проводятся через 3–4 нед. после ТУР. Вакцина для иммунотерапии рака мочевого пузыря**: 50–100 мг в 50мл физиологического раствора натрия хлорида**. Вводится еженедельно, в течение 6 нед, далее - ежемесячно на протяжении 1 года, либо по схеме: 3 недельные циклы каждые 3, 6, 12, 18, 24, 30, 36 мес. При БЦЖ-рефрактерных опухолях целесообразно выполнение радикальной ЦЭ.

Не рекомендуется проведение внутрипузырной инстилляции вакцины для иммунотерапии рака мочевого пузыря** в следующих случаях [205, 206]: 

в течение первых 2 недель после ТУР; 

пациентам с макрогематурией; 

после травматичной катетеризации; 

пациентам с наличием симптомов ИМП. 

Уровень убедительности рекомендаций – С (уровень достоверности доказательств – 5).

Комментарии: наличие лейкоцитурии или асимптоматической бактериурии не является противопоказанием для проведения БЦЖ-терапии, в этих случаях нет необходимости в проведении антибиотикопрофилактики. Системные осложнения могут развиться после системной абсорбции лекарственного препарата. Таким образом, следует учитывать противопоказания к внутрипузырной инстилляции [205, 206]. 

Рекомендуется с осторожностью проводить внутрипузырную БЦЖ-терапию пациентам для минимизации осложнений, вследствие большого количества побочных эффектов по сравнению с внутрипузырной химиотерапией [207, 208, 209, 210].  

Уровень убедительности рекомендаций – А (уровень достоверности доказательств – 2).

Комментарии: БЦЖ-терапия относительно противопоказана у иммунокомпрометированных пациентов (иммуносупрессия, ВИЧ-инфекция). Серьезные побочные эффекты встречаются менее чем у 5 \% пациентов и в большинстве случаев могут быть эффективно излечены. Показано, что поддерживающая схема лечения не ассоциирована с повышенным риском побочных эффектов в сравнении с индукционным курсом терапии. Некоторые небольшие исследования показали аналогичную эффективность и отсутствие увеличения количества осложнений по сравнению с не иммунокомпрометированными пациентами. В связи с тем, что БЦЖ-терапия слабо влияет на опухоли с низким риском развития рецидива, рекомендовано рассматривать ее как излишнее лечение для этой когорты пациентов [210]. 

Также отмечено, что у БЦЖ-терапии больше побочных эффектов, чем у ХТ. По этой причине оба вида лечения (БЦЖ-терапия и внутрипузырная ХТ противоопухолевыми антибиотиками) остаются возможными методами терапии. При окончательном его выборе следует учитывать риск рецидивирования и прогрессирования для каждого пациента в отдельности так же, как и эффективность и побочные эффекты любого метода лечения. 

В случае выявления БЦЖ-рефрактерной опухоли не рекомендовано дальнейшее консервативное лечение с применением вакцины

Альтернативой БЦЖ-терапии у отобранных больных может служить внутрипузырная химиотерапия. Остается спорным вопрос о продолжительности и частоте инстилляций химиопрепаратов. Из систематического обзора литературных данных по изучению РМП, где сравнивались различные режимы внутрипузырных инстилляций химиопрепаратов, можно сделать вывод, что идеальная продолжительность и интенсивность режимов остаются неопределенными из-за противоречивых результатов. Имеющиеся данные не подтверждают эффективность проведения лечения продолжительностью более 1 года [218]. 

Адаптация рН мочи, снижение дилюции с целью сохранения концентрации химиопрепарата снижают частоту рецидивов и являются важными условиями правильно проведенной инстилляции [216, 217]. При проведении внутрипузырной химиотерапии необходимо использовать лекарственные препараты при оптимальной рН мочи и поддерживать концентрацию препарата в течение экспозиции на фоне снижения потребления жидкости.

Схемы проведения внутрипузырной химиотерапии:

Вакцина для иммунотерапии рака мочевого пузыря**: 50–100 мг вакцины, разведенной в 50 мл физиологического раствора натрия хлорида**, вводится внутрипузырно на 2 часа с рекомендацией менять положение тела каждые полчаса. Доза 50 мг предназначена для пациентов с плохой индивидуальной пNCереносимостью терапии. Индукционный курс лечения проводится по схеме: еженедельно, в течение 6 нед. Поддерживающий курс лечения проводится по одной из схем: ежемесячно в течение 1 года или трехнедельные циклы каждые 3, 6, 12, 18, 24, 30, 36 мес.[247].

Митомицин**: 40 мг в 40 мл натрия хлорида**. Первая инстилляция – в течение 6 часов после выполнения ТУР, далее еженедельно, 6–8 инстилляций. Поддерживающий курс: ежемесячно, в течение 1 года. Экспозиция – 1–2 часа. [215].

Доксорубицин**: 30-50 мг в 25-50 мл 0,9 \% раствора натрия хлорида**. В случае развития местной токсичности (химический цистит) дозу следует растворить в 50-100 мл 0,9 \% раствора натрия хлорида**. Инстилляции можно проводить с интервалом от 1 недели до 1 месяца.

Внутрипузырная химиотерапия не проводится на протяжении более чем 1 года всем пациентам НМИ РМП вне зависимости от групп риска [219]. 

3.1.2.3. Фотодинамическая терапия
Рекомендуется фотодинамическая терапия как вариант 2 линии противоопухолевой терапии у пациентов с НМИ РМП при неэффективности предшествующего лечения [271].

Уровень убедительности рекомендаций – С (уровень достоверности доказательств – 4).

Комментарии: после внутривенного введения фотосенсибилизатора (cенсибилизирующего препарата, используемого для фотодинамической/лучевой терапии) с помощью лазера проводят обработку слизистой оболочки МП. В ряде работ сообщается об уменьшении количества рецидивов после фотодинамической терапии; в настоящее время осуществляются отработка схем и накопление материала. Дозы препаратов, сроки и режимы лечения зависят от распространенности опухоли по слизистой оболочке МП, характера фотосенсибилизатора и доз лазерного излучения.

3.1.2.4. Радикальная цистэктомия
Обоснованием радикальной цистэктомии как тактики лечения немышечно-инвазивного рака мочевого пузыря являются:

– несоответствие категории рТ1 после ТУР и последующей ЦЭ регистрируется у 27-51\% пациентов [137–140];

– худший прогноз у пациентов с прогрессией до МИ РМП, по сравнению первичным МИ РМП [141–142].

У пациентов с НМИ РМП выделяют срочную (незамедлительную) радикальную цистэктомию – сразу после установления диагноза РМП без инвазии в мышечный слой и раннюю радикальную цистэктомию – после неэффективной БЦЖ-терапии. Ретроспективно показано, что пациентам РМП с высоким риском развития рецидива лучше провести раннюю, чем отсроченную, ЦЭ при выявлении рецидива опухоли после первоначального лечения с использованием ТУР и БЦЖ-терапии, тем самым улучшая результаты выживаемости [127, 132, 143].

Необходимо учитывать влияние радикальной ЦЭ на качество жизни пациентов. Потенциальный положительный эффект от радикальной ЦЭ должен быть соизмеримым с возможными рисками и показателями заболеваемости.

Рекомендуется выполнение незамедлительной радикальной ЦЭ пациентам с НМИ РМП группы высочайшего риска для достижения ремиссии [127, 128,144].

Уровень убедительности рекомендаций – С (уровень достоверности доказательств – 3).

Комментарии: группа высочайшего риска включает пациентов со следующими характеристиками: уровень инвазии рТ1G3 с CIS; множественные, рецидивные опухоли больших размеров; pT1G3 с CIS в простатическом отделе уретры; редкие гистологические варианты опухоли с плохим прогнозом; опухоли Т1 с лимфоваскулярной инвазией. Эта группа прогностически наиболее неблагоприятная. При отказе пациента от ЦЭ показана БЦЖ-терапия с поддерживающим режимом в течение 3 лет.

При отказе или противопоказаниях к радикальной цистэктомии возможно проведение повторного курса терапии вакциной для иммунотерапии рака мочевого пузыря**. 

Рекомендуется выполнение ранней радикальной ЦЭ пациентам с БЦЖ-рефрактерными опухолями для достижения ремиссии [128].

Уровень убедительности рекомендаций – С (уровень достоверности доказательств – 3).

Комментарии: отсрочка в выполнении радикальной ЦЭ может привести к снижению показателей выживаемости. У пациентов с НМИ РМП после радикальной ЦЭ показатели 5-летней безрецидивной выживаемости превышают 80 \% [144–146].

3.1.2.5. Лечение пациентов с карциномой in situ
В случае неадекватного лечения более 50 \% пациентов с ранее выявленной CIS прогрессируют в мышечно-инвазивный (МИ) РМП [128]. Считается, что сочетание pТ1G2–3 и CIS имеет более худший прогноз по сравнению с первичной или распространенной CIS и CIS простатического отдела уретры [102, 129–131]. 

Рекомендуется проведение внутрипузырной иммунотерапии вакциной для лечения рака мочевого пузыря БЦЖ** всем пациентам с самостоятельной или сопутствующей CIS. Проведение внутрипузырной БЦЖ-терапии является важным прогностическим фактором и позволяет снизить риск прогрессирования с 66 до 20\% [132].

Уровень убедительности рекомендаций – С (уровень достоверности доказательств – 4).

Комментарии: БЦЖ-терапия должна включать индукционный курс и поддерживающий режим в течение 1–3 лет.

Рекомендуется проведение иммунотерапии пембролизумабом** 200 мг 1 раз в 3 недели или 400 мг 1 раз в 6 недель в/в капельно пациентам с CIS мочевого пузыря, резистентной к внутрипузырной БЦЖ-терапии, независимо от наличия папиллярной опухоли [304, 320, 336].

Уровень убедительности рекомендаций – С (уровень достоверности доказательств – 4).

Комментарии: пембролизумаб** изучался при БЦЖ-рефрактерной CIS мочевого пузыря в несравнительном исследовании в связи с отсутствием стандартного консервативного лечения, имеющего доказанную эффективность у подобной категории больных. В когорту А однорукавного многоцентрового исследования II фазы KEYNOTE-057 вошел 101 пациент с БЦЖ-рефрактерной CIS мочевого пузыря с или без папиллярной опухоли, имеющий противопоказания к радикальной цистэктомии или отказавшийся от нее. Гиперэкспрссия PD-L1 (≥10\% по шкале CPS) имела место в 38\% случаев. Всем пациентам проводилась монотерапия пембролизумабом** (200 мг 1 раз в 3 недели, в/в капельно) с оценкой эффекта каждые 3 месяца (цистоскопия, биопсия и цитологическое исследование мочи). Запланированная длительность лечения составляла 24 месяца. Терапию завершали преждевременно при выявлении персистирующего или рецидивного немышечно-инвазивного рака мочевого пузыря высокого риска, опухолевой прогрессии, метастазирования или при развитии непереносимой токсичности. Первичной целью исследования являлась частота объективных ответов. Частота полных ответов, зарегистрированных через 3 месяца терапии, составила 39\%. Медиана длительности полного ответа равнялась 16,2 месяца. При медиане наблюдения 36,4 месяца случаев опухолевой прогрессии в мышечно-инвазивный рак не зарегистрировано. Частота нежелательных явлений 3-4 степеней тяжести составила 13\% [320]. 


