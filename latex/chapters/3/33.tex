Схемы химиотерапии и иммунотерапии, применяемые при инвазивном и метастатическом РМП и используемые в данном разделе:

GC

гемцитабин** – 1000 мг/м2 в/в в 1-й, 8-й и 15-й день 

цисплатин** – 70 мг/м2 в/в в 1 (2)й день + гидратация - изотонический раствор натрия хлорида** (≈ 2,5л), с целью поддержания диуреза > 100 мл/ч в процессе введения цисплатина** и в последующие 3 ч [224].

Цикл повторяют каждые 4 нед.

GemCarbo

#гемцитабин** – 1000 мг/м2 в/в в 1-й и 8-й дни

карбоплатин** – AUC-4-5 в 1-й день (дозовый режим может быть изменен в зависимости от клинической ситуации - вынужденная редукция или эскалация дозы в пределах AUC-3-6)

Цикл повторяют каждые 3 нед [241].

MVAC

#винбластин** – 3 мг/м2 в/в во 2-й, 15-й, 22-й дни 

доксорубицин** – 30 мг/м2 в/в во 2-й день 

метотрексат** – 30 мг/м2 в/в в 1-й, 15-й, 22-й дни 

цисплатин** – 70 мг/м2 во 2-й день + гидратация 

Цикл повторяют каждые 4 нед [239].

DD-MVAC

#винбластин** – 3 мг/м2 в/в во 2-й, 

доксорубицин** – 30 мг/м2 в/в во 2-й день 

метотрексат** – 30 мг/м2 в/в в 1-й, 

#цисплатин** – 70 мг/м2 во 2-й день + гидратация 

рчГ-КСФ

Цикл повторяют каждые 2 нед [293].

MCV

#винбластин** – 4 мг/м2 в/в в 1-й, 8-й дни,

метотрексат** – 30 мг/м2 в/в в 1-й, 8-й дни 

цисплатин** – 100 мг/м2 во 2-й день + гидратация 

кальция фолинат** 15 мг в/в каждые 6 ч №4 во 2-й и 9-й дни

Цикл повторяют каждые 4 нед. [339]

Винфлунин – внутривенно медленно в течение 20 минут, по 320 мг/м2 каждые 3 недели (дозовый режим может быть изменен в зависимости от клинической ситуации - редукция дозы до 250-280 мг/м2)[243, 257, 309].

#Паклитаксел** 80 мг/м2 в 1, 8, 15-й дни; цикл 28 дней, или 175  мг/м2 в 1-й день; цикл 21 день [243, 257,309, 332].

#Доцетаксел** 75–100 мг/м2 в 1-й день; цикл 21 день [243, 257, 309, 333, 334].

#Гемцитабин** 1000–1250 мг/м2 в/в в 1, 8, 15-й дни; цикл 28 дней  [326-328, 335].

Иммуноонкологические препараты (моноклональные антитела):  

Атезолизумаб** – 840 мг в виде в/в инфузии каждые 2 недели, или 1200 мг в виде в/в инфузии каждые 3 недели, или 1680 мг в виде в/в инфузии каждые 4 недели. Первую дозу атезолизумаба** необходимо вводить в течение 60 минут. При хорошей переносимости первой инфузии все последующие введения можно проводить в течение 30 минут [288].   

Пембролизумаб** – 200 мг в виде в/в инфузии в течение 30 минут каждые 3 недели, или 400 мг в виде в/в инфузии в течение 30 минут каждые 6 недель [289, 294, 336].

Авелумаб** – 800 мг в виде в/в инфузии в течение 60 минут каждые 2 недели [304]

Ниволумаб** – 3 мг/кг или 240 мг в виде в/в инфузии каждые 2 недели, либо 480 мг в виде в/в инфузии каждые 4 недели. Первое введение должно быть осуществлено в течение 60 минут, при хорошей переносимости все последующие – на протяжении 30 минут [290, 291].

Оценка эффективности химиотерапии проводится на основании критериев ответа солидных опухолей на лечение (RECIST 1.1.). Оценка эффективности иммунотерапии проводится на основании критериев ответа солидных опухолей на лечение (iRECIST 1.1.). [приложение Г3].

Последовательность самостоятельной системной терапии первой и второй линий представлены в таблицах на стр. 116-117 [приложение Б].

3.3.1. Неоадъювантная химиотерапия
Применение только хирургического лечения обеспечивает 5-летнюю выживаемость лишь у 50\% пациентов МИ РМП [191, 220, 221]. С целью улучшения этих результатов более 30 лет применяется неоадъювантная платиносодержащая химиотерапия [222]. Несмотря на столь длительный период использования этого режима терапии, увеличение выживаемости не превышает 8\% [223].

Рекомендуется проведение неоадъювантной ХТ с включением схем на основе цисплатина** пациентам со стадией сТ2-Т4aN0/+M0 при наличии сохраненной функции почек (клиренс креатинина >60 мл/мин) и общего удовлетворительного состояния (ECOG <2) для уменьшения объема опухоли, воздействия на субклинические микрометастазы, повышения резектабельности опухоли и повышения выживаемости пациентов [223-228].

Уровень убедительности рекомендаций – А (уровень достоверности доказательств – 1).

Комментарии: терапию проводят перед хирургическим или лучевым лечением. Главное преимущество неоадъювантной ХТ – возможность оценить ее воздействие на первичный очаг, что может влиять на тактику дальнейшего лечения [145].

Рекомендуется использовать схемы неоадъювантной химиотерапии: GC, MVAC, DD-MVAC или MCV для увеличения выживаемости пациентов с МИ РМП и стадией сТ2-Т4аN0/+M0 [228, 293, 339] (расшифровка рекомендуемых схем дана в начале раздела 3.2.2. Химиотерапия).

Уровень убедительности рекомендаций – С (уровень достоверности доказательств – 4).

Комментарии: при использовании цисплатин**-содержащих схем, по разным данным, эффект был достигнут у 40-70\% пациентов. По результатам рандомизированных исследований продемонстрировано статистически значимое увеличение общей выживаемости на 5–8 \% среди получавших неоадъювантную ХТ [223-228].

Не рекомендуется всем пациентам с РМП проведение неоадъювантной ХТ в монорежиме [295].

Уровень убедительности рекомендаций – C (уровень достоверности доказательств – 4).

3.3.2. Адъювантная химиотерапия
В настоящее время продолжается дискуссия о целесообразности проведения адъювантной ПХТ у пациентов с высоким риском рецидива заболевания после радикальной (R0) [229]. Некоторые авторы считают, что адъювантная ХТ позволяет улучшить отдаленные результаты лечения в данной группе пациентов в среднем на 20-30\%. Однако вопрос о целесообразности адъювантного лечения, оптимальном режиме химиотерапии и о сроках ее проведения остается предметом клинических исследований. В настоящее время адъювантная ХТ по схемам: GC, MVAC или MCV, может быть рекомендована пациентам с рТ2–4N0/+М0R0, не получивших неоадъювантной ХТ [230-232].

Не рекомендуется рутинное применение адъювантной ХТ после хирургического лечения у пациентов с МИ РМП [229].

Уровень убедительности рекомендаций – В (уровень достоверности доказательств – 2).

Рекомендуется проведение адъювантной химиотерапии пациентам с МИ РМП стадии рТ2–4N0/+М0R0, соматически сохранным, способным перенести не менее 4 курсов химиотерапии после радикальной операции для увеличения продолжительности жизни [233-235].

Уровень убедительности рекомендаций – А (уровень достоверности доказательств – 2).

Комментарии: проводились рандомизированные исследования с применением различных схем адъювантной ХТ; в большинстве из них были получены данные о продлении безрецидивного периода по сравнению с контрольной группой (только радикальная ЦЭ) [233-235].

Рекомендуется проведение адъювантной терапии #ниволумабом** в дозе 240 мг 1 раз в 2 недели внутривенно в течение 1 года пациентам с МИ РМП со стадией рТ2–4N0/+М0R0, независимо от статуса PD-L1 и проведения НХТ [302, 303].

Уровень убедительности рекомендаций – А (уровень достоверности доказательств – 2).

Комментарий: по данным рандомизированного исследования III фазы CheckMate 274 (включившем 709 радикально оперированных больных уротелиальным раком группы высокого риска прогрессирования (T2-4 и/или N+)) адъювантная иммунотерапия #ниволумабом** достоверно увеличивает безрецидивную выживаемость, выживаемость без рецидива за пределами мочевыводящих путей, а также выживаемость без прогрессирования независимо от статуса PD-L1 и проведения неоадъювантной химиотерапии. Адъювантная иммунотерапия ассоциирована с благоприятным профилем безопасности и не ухудшает качество жизни пациентов [304].

3.3.3. Системная противоопухолевая терапия при неоперабельном местно-распространенном и метастатическом РМП
Первая линия лекарственной терапии

Выбор метода лекарственной терапии осуществляется на основании наличия противопоказаний к назначению цисплатина**, противопоказаний к назначению карбоплатина** и экспрессии PD-L1 в опухолевой ткани.

Рекомендуется пациентам с неоперабельным местно-распространенным и диссеминированным РМП, не имеющим противопоказаний к назначению цисплатина**, в первой линии терапии назначать химиотерапию в режимах GC или MVAC или DD-MVAC [234,239].

Уровень убедительности рекомендаций А (уровень достоверности доказательств – 2).

Комментарии: противопоказанием к назначению цисплатина** является наличие не менее одного из следующих критериев: соматический статус по классификации Eastern Cooperatve Oncology Group (ECOG) > 1; скорость клубочковой фильтрации (СКФ) ≤ 60 мл/мин/1,73 м2; снижение слуха ≥ 2 степени; периферическая нейропатия ≥ 2 степени или сердечная недостаточность класса III по классификации Нью-Йоркской кардиологической ассоциации [240].

В рандомизированном исследовании III фазы (n 405) больные уротелиальным раком IV стадии, не получавшие предшествующей терапии, были рандомизированы на ХТ по схеме GC или M-VAC. Режимы продемонстрировали сопоставимые частоту объективного ответа, время до прогрессирования и 18-месячную общую выживаемость. Наиболее значимыми видами токсичности являлись миелотоксичность, сепсис на фоне фебрильной нейтропении и мукозит. У больных, получавших GC, чаще отмечались тяжелая анемия и тромбоцитопения; в группе, получавшей M-VAC, чаще регистрировались тяжелая, фебрильная нейтропения, а также тяжелые мукозиты [234].

Крупное рандомизированное исследование фазы III сравнивало DD-MVAC с поддерживающей терапией гранулоцитарными колониестимулирующими факторами со стандартным MVAC. DD-MVAC увеличивал частоту объективного ответа, однако не приводил к значимому увеличению медианы общей выживаемости. У пациентов, получавших DD-MVAC c гранулоцитарными колониестимулирующими факторами, наблюдалась меньшая общая токсичность [239].

Рекомендуется выполнять патолого-анатомическое исследование биопсийного (операционного) материала с применением иммуногистохимической оценки экспрессии PD-L1 всем пациентам с неоперабельным местно-распространенным и диссеминированным РМП в первой линии терапии, которые имеют противопоказания к назначению цисплатина**. При планировании терапии пембролизумабом** оценка экспрессии PD-L1 должна производиться по шкале CPS с использованием клона 22С3 [299,304], атезолизумабом** – по шкале IC с использованием клона SP 142 [244,305].

Уровень убедительности рекомендаций С (уровень достоверности доказательств – 4).

Комментарии: основанием для регистрации ингибиторов PD-(L)1 для первой линии терапии распространенного уротелиального рака у пациентов с противопоказаниями к цисплатину**, послужили исследования II фазы, в которых применялось PD-L1 тестирование опухолевой ткани. В исследовании пембролизумаба** использовалась комбинированная шкала оценки экспрессии PD-L1 (CPS), учитывающая позитивные клетки опухоли и клетки иммунной системы, инфильтрирующие опухоль [299]; в исследовании атезолизумаба** учитывалось окрашивание только иммунных клеток [244]. Результаты применения данных препаратов при оценке экспрессии по иным шкалам не изучались. В связи с этим, для селекции кандидатов для иммунотерапии пембролизумабом** и атезолизумабом** необходимо тестирование с использованием шкал с доказанной предикторной ценностью. 

Рекомендуется пациентам с неоперабельным местно-распространенным и диссеминированным РМП, имеющим противопоказания к назначению цисплатина** и гиперэкспрессию PD-L1 в опухолевой ткани, проведение иммунотерапии:

- при гиперэкспрессии PD-L1 ≥10\% - монотерапии пембролизумабом** (200 мг в виде в/в инфузии в течение 30 минут каждые 3 недели или 400 мг 1 раз в 6 недель) [299, 304, 336];

- при гиперэкспрессии PD-L1 ≥5\% – монотерапии атезолизумабом** (840 мг в виде в/в инфузии каждые 2 недели, или 1200 мг в виде в/в инфузии каждые 3 недели, или 1680 мг в виде в/в инфузии каждые 4 недели) [244, 305, 337].

Уровень убедительности рекомендаций В (уровень достоверности доказательств – 3).

Комментарии: эффективность и безопасность пембролизумаба** в первой линии терапии распространенного уротелиального рака изучались в рамках многоцентрового исследования II фазы KEYNOTE-052, включившего 374 больных, имевших противопоказания к терапии цисплатином**. Первичной целью являлась частота объективного ответа у всех пациентов и у больных с гиперэкспрессией PD-L1. Оценка PD-L1-статуса проводилась по CPS. Пограничное значение экспрессии PD-L1 было выделено у первых 100 больных и составило 10\%. Частота объективного ответа у всех больных составила 24\%, у пациентов с экспрессией PD-L1 ≥10\% - 38\%. Медиана времени до ответа равнялась 2 месяца, при медиане наблюдения 5 месяцев 83\% ответов продолжались, медиана длительности ответа не достигнута. Наиболее распространенными нежелательными явлениями 3-4 степени тяжести, связанными с лечением, являлись слабость (2\%), повышение уровня сывороточной щелочной фосфатазы (1\%) и снижение мышечной силы (1\%) [299].

Ингибитор PD-L1 атезолизумаб** в первой линии терапии распространенного уротелиального рака у больных с противопоказаниями к терапии цисплатином** изучался в 1 когорте исследования IMvigor210. Статус экспрессии PD-L1 на инфильтрирующих лимфоцитах в микроокружении опухоли определяли как процент позитивных иммунных клеток: IC0 (<1\%), IC1 (≥1\% но <5\%) и IC2/3 (≥5\%). Первичной целью являлась частота объективного ответа, которая составила 23\% у всех пациентов и достигла 28\% у больных с гиперэкспрессией PD-L1 IC2/3. При медиане наблюдения 17,2 мес медиана длительности ответа не достигнута. Нежелательные явление, связанные с лечением, наблюдались у 66\% (3-4 степени тяжести – у 16\%) больных [244].

Рекомендуется пациентам с неоперабельным местно-распространенным и диссеминированным РМП, имеющим противопоказания к назначению цисплатина** и отсутствие гиперэкспрессии PD-L1 в опухолевой ткани, проведение химиотерапии в режиме GemCarbo [241].

Уровень убедительности рекомендаций А (уровень достоверности доказательств – 2).

Комментарии: рандомизированное исследование II/III фазы EORTC 30986 сравнивало две схемы, содержащие карбоплатин** (метотрексат**, карбоплатин**, винбластин** (M-CAVI) и GemCarbo, у пациентов с такими противопоказаниями к цисплатину**, как СКФ <60 мл/мин/1,73 м2 и/или соматический статус ECOG 2. Оба режима продемонстрировали противоопухолевую активность: частота объективного ответа составила 42\% для GemCarbo и 30\% для M-CAVI. Частота тяжелых нежелательных явлений достигла 13,6\% и 23\% в группах исследования, соответственно [241]. На основании этих данных комбинация GemCarbo стала стандартом лечения этой группы пациентов.

Рекомендуется пациентам неоперабельным местно-распространенным или диссеминированным уротелиальным раком мочевого пузыря, достигшим контроля над опухолью (полный, частичный ответ или стабилизация опухолевого процесса) после 4-6 циклов химиотерапии, основанной на препаратах платины, проведение поддерживающей терапии авелумабом**. [301,302]

Уровень убедительности рекомендаций – А (уровень достоверности доказательств – 2).

Комментарии: рандомизированное клиническое исследование III фазы JAVELIN Bladder 100 изучало влияние поддерживающей терапии ингибитором PD-L1 авелумабом** после первой линии лечения комбинацией препарата платины и гемцитабина** у больных распространенным уротелиальным раком с объективным ответом или стабилизацией опухолевого процесса после 4–6 циклов химиотерапии.  Больных рандомизировали в группу авелумаба** или наилучшей поддерживающей терапии. Авелумаб** значимо увеличивал общую выживаемость с 14,3 до 21,4 месяца (HR: 0,69; 95\% СI: 0,56–0,86; p <0,001). Нежелательные явления ≥3 степени тяжести наблюдались у 47\% больных группы авелумаба** по сравнению с 25\% пациентов группы контроля. Иммуно-опосредованные нежелательные явления отмечены в 29\% случаев, достигли ≥3 степени тяжести у 7\% больных и включали колит, пневмонит, сыпь, повышение уровня печеночных ферментов, гипергликемию, миозит и гипотиреоз [301, 302].

Рекомендуется пациентам с неоперабельным местно-распространенным и диссеминированным РМП, не имеющим противопоказаний к назначению препаратов платины, в первой линии терапии назначать иммуно-химиотерапию гемцитабином** с препаратом платины и #атезолизумабом** независимо от экспрессии PD-L1 [300].

- пациентам без противопоказаний к цисплатину**:

#гемцитабин** – 1000 мг/м2 в/в в 1-й и 8-й дни

цисплатин** – 70 мг/м2 в/в в 1-й (2-й) день + гидратация - изотонический раствор натрия хлорида** (≈ 2,5 л), с целью поддержания диуреза > 100 мл/ч в процессе введения цисплатина** и в последующие 3 ч.

Цикл повторяют каждые 3нед.

#атезолизумаб** – 1200 мг в/в капельно, каждые 3 недели.

- пациентам с противопоказаниями к цисплатину**:

#гемцитабин** – 1000 мг/м2 в/в в 1-й и 8-й дни

карбоплатин** – AUC-4,5 в 1-й день

Цикл повторяют каждые 3 нед

#атезолизумаб** – 1200 мг в/в капельно, каждые 3 недели

Уровень убедительности рекомендаций А (уровень достоверности доказательств – 2).

Комментарии: рандомизированное исследование IMvigor130 сравнивало комбинацию ингибитора PD-L1 атезолизумаба** с ХТ в режимах GC/GemCarbo с ХТ GC/GemCarbo в сочетании с плацебо или монотерапией атезолизумабом**. В исследовании была достигнута первичная конечная точка: иммуно-химиотерапия обеспечивала преимущество беспрогрессивной выживаемости по сравнению с химиотерапией и плацебо во всей популяции больных (8,2 и 6,3 месяца, соответственно; HR: 0,82 (95\% CI: 0,70–0,96); p = 0,007). Незрелые данные по ОВ при медиане наблюдения 11,8 месяца не продемонстрировали различий между группами. Из-за иерархического дизайна тестирования сравнение ХТ с монотерапией атезолизумабом** еще не проводилось [300].

Рекомендуется пациентам с неоперабельным местно-распространенным и диссеминированным РМП, имеющим противопоказания к назначению карбоплатина** проведение иммунотерапии независимо от гиперэкспрессии PD-L1 в опухолевой ткани:

- монотерапии пембролизумабом** (200 мг в виде в/в инфузии в течение 30 минут каждые 3 недели или 400 мг 1 раз в 6 недель) [299, 304, 337];

- монотерапии атезолизумабом** (840 мг в виде в/в инфузии каждые 2 недели, или 1200 мг в виде в/в инфузии каждые 3 недели, или 1680 мг в виде в/в инфузии каждые 4 недели) [244, 305].

Уровень убедительности рекомендаций В (уровень достоверности доказательств – 3).

Комментарии: эффективность и безопасность #пембролизумаба** в первой линии терапии распространенного уротелиального рака изучались в рамках многоцентрового исследования II фазы KEYNOTE-052, включившего 374 больных, имевших противопоказания к терапии цисплатином**. Частота объективного ответа у всех больных составила 24\%, у пациентов с экспрессией PD-L1 ≥10\% - 38\%. Медиана времени до ответа равнялась 2 месяца, при медиане наблюдения 5 месяцев 83\% ответов продолжались, медиана длительности ответа не достигнута. Наиболее распространенными нежелательными явлениями 3-4 степени тяжести, связанными с лечением, являлись слабость (2\%), повышение уровня сывороточной щелочной фосфатазы (1\%) и снижение мышечной силы (1\%) [299].

Ингибитор PD-L1 атезолизумаб** в первой линии терапии распространенного уротелиального рака у больных с противопоказаниями к терапии цисплатином** изучался в 1 когорте исследования IMvigor210. Первичной целью являлась частота объективного ответа, которая составила 23\% у всех пациентов. При медиане наблюдения 17,2 мес медиана длительности ответа не достигнута. Нежелательные явление, связанные с лечением, наблюдались у 66\% (3-4 степени тяжести – у 16\%) больных [244].

Рекомендовано пациентам с неоперабельным местно-распространенным и диссеминированным РМП, имеющим противопоказания к назначению карбоплатина**, проведение монохимиотерапии препаратами других фармакологических групп (#доцетаксел**, #паклитаксел**, #гемцитабин**) [241, 306, 307, 324-328, 338].

Уровень убедительности рекомендаций С (уровень достоверности доказательств – 4).

Комментарии: формально противопоказанием к применению карбоплатина** является выраженное снижение функции костного мозга. Однако в клинической практике у больных распространенным уротелиальным раком в качестве факторов, исключающих возможность назначения карбоплатина**, используются критерии, заимствованные из рандомизированного исследования EORTC 30986 (низкий соматический статус ECOG >2, СКФ <30 мл/мин/1,73 м2 или комбинация соматического статуса ECOG 2 и СКФ <60 мл/мин/1,73 м2), так как прогноз этой популяции пациентов плохой независимо от проведения ХТ на основе препаратов платины или без них [241]. Данные о возможностях лекарственного противоопухолевого лечения у данной группы пациентов ограничены отдельными однорукавными исследованиями, показавшими приемлемую эффективность и безопасность монотерапии таксанами [306, 307] и гемцитабином** [324-328]. Имеющейся доказательной базы недостаточно для формирования клинических рекомендаций.

Вторая линия лекарственной терапии

Рекомендуется в качестве режима предпочтения назначение монотерапии пембролизумабом** больным неоперабельным местно-распространенным или метастатическим РМП с прогрессированием после или на фоне проведения химиотерапии, основанной на препаратах платины [257].

Уровень убедительности рекомендаций А (уровень достоверности доказательств – 2).

Комментарии: рандомизированное исследование III фазы KEYNOTE-045 было направлено на сравнение эффективности пембролизумаба** и традиционной химиотерапии у больных неоперабельным местно-распространенным и диссеминированным уротелиальным раком, прогрессирующим на фоне или в течение 12 месяцев после завершения ХТ, основанной на цисплатине**. В исследование было включено 542 пациента, рандомизированного на терапию пембролизумабом** или монохимиотерапию (паклитаксел**, доцетаксел** или винфлунин). Первичной целью являлась оценка общей и беспрогрессивной выживаемости во всей популяции исследования и у больных с экспрессией PD-L1 ≥10\% по CPS. При медиане наблюдения 18,5 месяца пембролизумаб** значимо увеличивал медиану общей выживаемости с 7,4 до 10,3 месяца (HR 0,70; 95\% CI:0,57-0,86; p = 0,0004). Различия беспрогрессивной выживаемости между группами были недостоверны (медиана – 2,1 в группе пембролизумаба** vs 3,3 месяца в группе химиотерапии, 18-месячная – 16,8\% vs 3,5\% соответственно; р=0,32). Частота объективного ответа и полного ответа в группе пембролизумаба** составила 21,1\% и 7,8\%, в группе химиотерапии – 11,0\% и 2,9\% соответственно. Медиана длительности ответа на фоне терапии пембролизумабом** не достигнута, на фоне химиотерапии – 4,4 месяца. Наличие экспрессии PD-L1 (CPS≥10\%) не оказывало влияния на частоту объективного ответа и показатели выживаемости. Иммунотерапия лучше переносилась пациентами: любые нежелательные явления, связанные с лечением, зарегистрированы у 61,3\% больных в группе пембролизумаба** и у 90,2\% пациентов, получавших химиотерапию; токсичность ≥ 3 степени тяжести зарегистрирована у 16,5\% и 49,8\% больных соответственно [257].

Рекомендуется в качестве альтернативного режима назначение монотерапии ниволумабом** больным неоперабельным местно-распространенным или метастатическим РМП с прогрессированием после или на фоне проведения химиотерапии, основанной на препаратах платины [246,308].

Уровень убедительности рекомендаций С (уровень достоверности доказательств – 4).

Комментарии: ниволумаб** был изучен в качестве монотерапии при диссеминированном уротелиальном раке в исследовании I/II фазы CheckMate 032 у пациентов, получавших не менее 1 предшествующей линии лечения, включавшего препараты платины, независимо от статуса PD-L1. Первичной целью исследования являлась частота объективного ответа, которая составила 24,4\% и не зависела от уровня экспрессии PD-L1. Связанные с лечением нежелательные явления 3-4 степени тяжести развились у 22\% пациентов; наиболее частыми из них были повышение сывороточной липазы (5\%) и амилазы (4\%) [308].

В исследовании II фазы Checkmate 275 (n 270) ниволумаб** во второй линии терапии резистентного к препаратам платины уротелиального рака позволил добиться объективного ответа в 19,6\% случаев при медиане времени до лечебного эффекта 1,9 мес. Частота объективного ответа нарастала по мере увеличения уровня экспрессии PD-L1 и составила 28,4\% при положительном окрашивании ≥5\% клеток, и 16,1\% у пациентов с экспрессией PD-L1 <5\%. Медиана беспрогрессивной выживаемости составила 2 месяца (1,87 месяца - при экспрессии PD-L1 <1\% и 3,6 месяца - ≥1\%). Медиана общей выживаемости равнялась 8,74 месяца у всех больных (5,95 месяца – при экспрессии PD-L1 <1\% и 11,3 месяца - при экспрессии PD-L1 ≥1\%). НЯ 3-4 степени тяжести, связанные с лечением, имели место в 18\% наблюдений [246].

Рекомендуется в качестве альтернативного режима назначение монотерапии атезолизумабом** больным неоперабельным местно-распространенным или метастатическим РМП с прогрессированием после или на фоне проведения химиотерапии, основанной на препаратах платины или препаратах других фармакологических групп [301,310,311].

Уровень убедительности рекомендаций С (уровень достоверности доказательств – 4).

Комментарии: терапия атезолизумабом** при резистентных опухолях изучалась во 2 когорте исследования IMvigor 210, включившей 315 больных распространенным уротелиальным раком, ранее получавших препараты платины. Первичной целью исследования являлась оценка частота объективного ответа, которая составила 16\% у всех больных и достигла 28\% при PD-L1 IC2/3 [310].

IMvigor211 – рандомизированное исследование III фазы, сравнивавшее эффективность и безопасность атезолизумаба** и ХТ (винфлунин, паклитаксел** или доцетаксел**) у пациентов диссеминированным уротелиальным раком, в течение или после как минимум одного цитотоксического режима терапии, основанного на препаратах платины (n 931). Исследование было отрицательным: достоверных различий общей выживаемости всей популяции пациентов, получавших атезолизумаб** или химиотерапию, не выявлено (медиана - 8,6 vs. 8,0 месяца соответственно, HR 0,85; 95\% CI: 0,73 – 0,99) [301,309].

В исследовании IIIb фазы SAUL эффективность и безопасность атезолизумаба** изучались у 1004 пациентов резистентным местно-распространенным или метастатическим уротелиальным или неуротелиальным раком мочевыводящих путей, включая больных, не соответствующих рутинным критериям включения в клинические исследования, в том числе - пациентов, получавших химиотерапию, основанную не на препаратах платины. Медиана общей выживаемости составила 8,7 месяца, медиана беспрогрессивной выживаемости - 2,2 месяца, частота объективного ответа - 13\%. Нежелательные явления ≥3 степени зарегистрированы у 45\% пациентов, что привело к прекращению лечения из-за токсичности в 8\% случаев [311].

Рекомендуется в качестве альтернативного режима назначение монотерапии винфлунином больным неоперабельным местно-распространенным или метастатическим РМП с прогрессированием после или на фоне проведения химиотерапии, основанной на препаратах платины [242].

Уровень убедительности рекомендаций С (уровень достоверности доказательств – 4).

Комментарии: в рандомизированное исследование III фазы, сравнивавшее винфлунин с наилучшей поддерживающей терапией при распространенном уротелиальном раке у больных с прогрессированием после проведения химиотерапии, основанной на цисплатине**, вошло 370 больных. Винфлунин продемонстрировал недостоверное преимущество общей выживаемости во всей популяции пациентов по сравнению с поддерживающим лечением (6,9 vs. 4,6 месяца соответственно, HR 0,88; 95\% CI, 0,69-1,12: P = 0,287). Однако при анализе фактических лечебных групп разница результатов в пользу винфлунина оказалась статистически значимой в отношении общей выживаемости (6,9 vs. 4,3 соответственно, P = 0,04), а также частоты объективного ответа (16\% vs 0\%, P = 0,0063), контроля над болезнью (41,1\% vs 24,8\%, P = 0,0024) и медианы БПВ (3,0 vs 1,5 месяца, P = 0,0012). Длительность объективного ответа на терапию винфлунином составила 7,4 месяца (95\% CI 4,5 – 17,0 месяца) [242].

Рекомендуется в качестве альтернативного режима назначение монотерапии #гемцитабином** или #паклитакселом** или #доцетакселом** больным неоперабельным местно-распространенным или метастатическим РМП с прогрессированием после или на фоне проведения химиотерапии, основанной на препаратах платины [243, 257, 309, 338].

Уровень убедительности рекомендаций С (уровень достоверности доказательств – 4).

Комментарии: IMvigor211 – рандомизированное исследование III фазы, сравнивавшее эффективность и безопасность атезолизумаба** и ХТ (винфлунин, паклитаксел** или доцетаксел**) у пациентов диссеминированным уротелиальным раком, в течение или после как минимум одного цитотоксического режима терапии, основанного на препаратах платины (n 931). Монохимиотерапия не уступала иммунотерапии в отношении общей выживаемости у всей популяции пациентов (медиана - 8,6 vs. 8,0 месяца соответственно, HR 0,85; 95\% CI: 0,73 – 0,99) [301,309].

Рекомендуется в качестве альтернативного режима назначение монотерапии #гемцитабином** больным неоперабельным местно-распространенным или метастатическим РМП с прогрессированием после или на фоне проведения химиотерапии, основанной на препаратах платины [325-328].

Уровень убедительности рекомендаций С (уровень достоверности доказательств – 4).

Комментарии: активность #гемцитабина** при распространенном РМП была продемонстрирован в ходе испытания I фазы, в котором четыре (27\%) из 15 пациентов, ранее получавших M-VAC, ответили на лечение. Дозы гемцитабина варьировали от 875 мг/м2 до 1307 мг/м2 в 1, 8 и 15 дни 28-дневного цикла [325]. Три последующие исследования фазы II с использованием #гемцитабина** в дозе 1200 мг/м2 в 1, 8 и 15 дни 4-недельного цикла продемонстрировали частоту ответов в диапазоне от 23\% до 28\% [326-328].


