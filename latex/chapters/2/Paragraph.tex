\section{Диагностика заболевания или состояния (группы заболеваний или состояний) медицинские показания и противопоказания к применению методов диагностики}
\label{sec:Method}
Критерии установления диагноза/состояния:

Данные анамнеза.

Данные физикального обследования.

Данные лабораторных исследований.

Данные инструментального обследования.

Данные патолого-анатомического исследования.

Клинический диагноз основан на следующих результатах обследования:

Физикальный осмотр, данные анамнеза (макрогематурия) позволяют заподозрить новообразование мочевого пузыря.

Лабораторные исследования могут выявить наличие эритроцитов в моче.

Применение цистоскопии наиболее полно позволяет оценить состояние полости мочевого пузыря.

Заключение патолого-анатомического исследования опухолевого материала (биопсия новообразований).

Данные лучевых методов диагностики позволяют корректно стадировать заболевание.

\subsection{Жалобы и анамнез}
\label{sec:}
Жалобы и анамнез описаны в разделе «Клиническая картина».


\subsection{Физикальное обследование}
\label{sec:}
Рекомендуется всем пациентам проводить физикальное обследование для оценки общего состояния пациента [58–60].

Уровень убедительности рекомендаций – С (уровень достоверности доказательств – 4).

Комментарии: физикальное обследование включает в себя бимануальную ректальную и вагинальную пальпацию. Пальпируемая опухолевая масса может быть выявлена у пациентов с местно-распространенными опухолями. Во время наркоза, до и после проведения ТУР МП также целесообразно бимануальное исследование, чтобы оценить, имеется ли пальпируемая масса и фиксирована ли опухоль к стенке таза [58, 59]. Однако, учитывая несоответствие между бимануальным исследованием и стадией pT после цистэктомии (ЦЭ) (11 \% клинической переоценки и 31 \% недооценки), при интерпретации данных бимануального исследования рекомендуется соблюдать определенную осторожность [60].

При массивной гематурии имеются проявления анемии – бледность кожных покровов, слабость, вялость.

Рекомендуется всем пациентам при физикальном осмотре выполнить пальпацию мочевого пузыря, области почек с определением симптома поколачивания; проведение тщательного осмотра и пальпации зон возможного лимфогенного метастазирования для верификации диагноза [58–60].

Уровень убедительности рекомендаций – С (уровень достоверности доказательств – 4).

Комментарии: при немышечно-инвазивных формах РМП бимануальные манипуляции нецелесообразны. В случае тампонады мочевого пузыря, острой задержки мочи переполненный МП пальпируется над лоном, а пациента беспокоят постоянные позывы на мочеиспускание, ощущение распирания в проекции мочевого пузыря, боль в надлобковой области. В то же время при увеличении объема опухоли снижается емкость МП, нарушается его резервуарная функция, что проявляется постоянными позывами на мочеиспускание с небольшим количеством выделяемой мочи, частичным недержанием. При блоке опухолью устья мочеточника и развитии гидронефроза положителен симптом поколачивания, пальпируется увеличенная почка. При местно-распространенном процессе информативна бимануальная пальпация МП, которая позволяет оценить размеры, подвижность опухоли, наличие инфильтрации окружающих тканей.

\subsection{Лабораторные диагностические исследования}
\label{sec:}

Рекомендуется всем пациентам проведение цитологического исследования мочи (исследования мочи для выявления клеток опухоли) или промывных вод из полости МП перед выполнением трансуретральной резекции (ТУР) для верификации диагноза [61–64].

Уровень убедительности рекомендаций – В (уровень достоверности доказательств – 2).

Комментарии: наиболее адекватным материалом является взятие промывных вод при цистоскопии. ЦИ мочи имеет высокую чувствительность при T1 и G3 (84 \%), в случае Та и G1 – низкую (16 \%) [61]. Чувствительность при CIS составляет 28–100 \% [62] (УД 1В). Данную методику необходимо использовать в качестве дополнения к цистоскопии. Тем не менее, стоит отметить, что положительные результаты цитологического исследования могут указывать на наличие опухоли в любом отделе мочевыводящих путей [63]. Однако, негативные результаты не исключают наличие опухоли [64].

Рекомендуется всем пациентам проводить патолого-анатомическое исследование биопсийного (операционного) материала, полученного с помощью ТУР для верификации диагноза [81–83].

Уровень убедительности рекомендаций – С (уровень достоверности доказательств – 5).

Комментарии: патоморфологическое исследование образца является важным в диагностике и лечении РМП. Требуется тесное сотрудничество между врачами-хирургами и врачами-патологоанатомами. Высокое качество предоставленной ткани и клиническая информация необходимы для правильной диагностической оценки. Наличие мышечной ткани в материале необходимо для правильного установления категории Т [82]. В сложных случаях следует рассмотреть вопрос о дополнительном пересмотре материала опытным врачом-патологоанатомом.

Важно наличие в направлении на патологоанатомическое исследование анамнестических и клинических данных: наличие рецидивов, химиотерапии, лучевой терапии, БЦЖ-терапии в анамнезе, локализация опухоли, уни- или мультицентрическое поражение.

В патологоанатомическом заключении после трансуретральной резекции мочевого пузыря следует указывать:

− гистологический тип опухоли;

− процент гетерологической дифференцировки (плоскоклеточной/ железистой/трофобластической и т.п.) и/или специфического подтипа уротелиальной карциномы при наличии

− гистологическую степень злокачественности опухоли (грейд)

− наличие инвазии в субэпителиальную строму и мышечный слой стенки

− наличие или отсутствие мышечного слоя

− наличие лимфоваскулярной инвазии

− наличие неинвазивной опухоли, карциномы in situ

− стадию по ТNM

− гистологический код по МКБ-О

Время холодовой ишемии (промежуток времени от прекращения кровообращения в органе до его адекватной фиксации) не должно превышать 2 часов.  Операционный материал рекомендуется предварительно фиксировать в 10\% нейтральном формалине в течение 10–12 часов перед забором фрагментов в гистологические кассеты (вырезкой). К предварительной фиксации образец необходимо подготовить: вскрыть просвет мочевого пузыря по передней стенке Т- или Y-образным разрезом от устья уретры и/или наполнить просвет мочевого пузыря формалином, перед его погружением в достаточный  объем фиксирующей жидкости (в 10–20 раз превышающий объем образца). При макроскопическом исследовании оценивается максимальный размер опухоли и протяженность/глубина инвазии. Все доставленные с мочевым пузырем органы и их фрагменты (единым блоком/ или отдельно) должны быть описаны, измерены и исследованы гистологически.

В патологоанатомическом заключении после цистэктомии следует указывать:

- гистологический тип опухоли

- наличие гетерологической дифференцировки (плоскоклеточной/ железистой/ трофобластической и т.п.) и/или специфического подтипа уротелиальной карциномы и их долю от опухоли

- наличие неинвазивной опухоли, карциномы in situ

- гистологическую степень злокачественности опухоли (грейд)

- глубину инвазии (по данным микроскопического исследования)

-  наличие лимфоваскулярной инвазии

- статус краев резекции

- количество удаленных и метастатически измененных лимфатических узлов

- стадию TNM

- гистологический код по МКБ-О.

Для патологоанатомического исследования биопсийного (операционного) материала необходимо использовать классификацию ВОЗ 2016 г. Необходимо указывать стадию и степень злокачественности опухолевого процесса при исследовании каждый раз, когда используется термин «немышечно-инвазивный РМП» [83]. Использовать термин «поверхностный РМП» не рекомендуется.

\subsection{Инструментальные диагностические исследования}
\label{sec:}
Цистоскопия

Диагностика папиллярных форм РМП напрямую зависит от цистоскопического исследования и гистологического заключения на основании холодовой, щипковой биопсии или ТУР-биопсии. В случае подозрения на CIS обязательным является также цитологическое исследование и выполнение множественной рандомной биопсии стенок МП [84].

Рекомендуется проведение цистоскопии (стандартной или флюоресцентной/фотодинамической) всем пациентам при подозрении на РМП для верификации диагноза [26, 85-99].

Уровень убедительности рекомендаций – А (уровень достоверности доказательств – 1).

Комментарии: Цистоскопия может быть проведена как амбулаторная процедура. С помощью гибкого цистоскопа с трансуретральным введением местного анестетика достигается лучшая переносимость, особенно у мужчин. Для предупреждения пропуска опухоли должен быть выполнен тщательный осмотр всего эпителия, выстилающего полость МП. Если опухоль МП визуализирована при ранее выполненных визуальных методах исследования, диагностическую цистоскопию можно не выполнять, так как этим пациентам показано проведение ТУР [85, 86].

Диагностическая цистоскопия не должна заменять цитологическое исследование или какие-либо другие неинвазивные методы. При цистоскопии рекомендовано описать все макроскопические характеристики опухоли [87]:

локализация;

размер;

количество;

внешний вид;

характер роста (экзофитный, эндофитный, смешанный);

патологические участки слизистой оболочки;

слизистая оболочка мочеиспускательного канала.

Рекомендуется использовать схему строения МП [88]. При проведении диагностической уретроцистоскопии и выявлении новообразования рекомендована биопсия либо выполнение ТУР-биопсии.

Использование флюоресцентной цистоскопии позволяет более точно проводить исследование и более четко определять границы измененных участков, особенно при CIS [89, 90] (УД 1). ФД проводится в фиолетовом свете после внутрипузырной экспозиции фотосенсибилизатора аминолевулиновой кислоты (cенсибилизирующий препарат, используемый для фотодинамической/лучевой терапии). Чувствительность ФД составляет 80–96\%, специфичность – 65–70 \% [90, 91]. Такие факторы, как воспалительный процесс, ТУР, проведенная в ближайшие 3 мес., БЦЖ-терапия, снижают качество метода, увеличивая ложноположительные результаты [92, 93]. Другими исследованиями было подтверждено, что в опытных руках частота ложноположительных ответов при ФД была сопоставима с результатами цистоскопии в белом свете [26, 94].

Альтернативным вариантом может быть узкоспектральная визуализация (narrow band imaging, NBI), не требующая введения cенсибилизирующих препаратов, используемых для фотодинамической/лучевой терапии. Благодаря специальным фильтрам увеличивается контрастность между неизмененной слизистой и гиперваскулярной опухолевой тканью [95]. Когортные проспективные исследования продемонстрировали преимущество данного метода при выявлении рецидивов, а также более прецизионное выполнение на его фоне ТУР [96–98]. Однако рандомизированные исследования не показали достоверной разницы в выявлении рецидивов, а также преимуществ данного метода при оценке прогрессирования и смертности [99].

Биопсия

При визуализации подозрительных участков, в случае положительной цитологии, при динамическом контроле, особенно если в анамнезе были ТУР по поводу CIS или T1G3, – во всех случаях показана холодовая биопсия как отдельных участков, так и всех стенок МП [100, 101].

Рекомендуется пациентам выполнение случайной множественной биопсии для оценки распространения опухолевого процесса при T1G3/CIS из следующих участков МП [102-105]:

− треугольник Льето;

− верхушка мочевого пузыря;

− правая, левая, передняя и задняя стенки МП;

− простатический отдел уретры.

Уровень убедительности рекомендаций – С (уровень достоверности доказательств – 4).

Комментарии: Биопсия простатического отдела уретры особенно целесообразна в случаях, когда имеется CIS, локализованная в области треугольника или детрузора, а также при высоком риске T1G3. Представленные Palou и соавт. результаты продемонстрировали, что у пациентов с T1G3 частота обнаружения сопутствующей CIS в простатической части уретры составила 11,7 \% [102, 103].

ТУР-биопсия является наиболее информативным вариантом биопсии. Пациенту проводится трансуретральное удаление части опухоли мочевого пузыря с подлежащим мышечным слоем и последующей оценкой глубины инвазии [104]. Также при ТУР-биопсии можно более точно оценить первичную опухоль, а именно ее вертикальные и горизонтальные размеры, используя при этом резекционную петлю [105]. Ширина петли составляет до 1 см.

Ультразвуковая диагностика

Рекомендуется проведение ультразвукового исследования (УЗИ) почек и МП пациентам с гематурией в качестве первичной диагностики. УЗИ дает возможность оценить расположение, размеры, структуру, характер роста, распространенность опухоли, измерить емкость мочевого пузыря, оценить деформацию стенок. Во время проведения исследования также возможно визуализировать зоны регионарного метастазирования, верхние мочевыводящие пути (ВМП), наличие или отсутствие гидронефроза. Пациентам с подтвержденным РМП рекомендуется проведение УЗИ органов брюшной полости и забрюшинного пространства [106].

Уровень убедительности рекомендаций – С (уровень достоверности доказательств – 5).

Комментарии: УЗИ проводят трансабдоминально, трансректально (у женщин –трансвагинально) при наполненном МП. Необходимо выполнять УЗИ печени и парааортальной зоны при высокой вероятности распространенного процесса. Информативность метода резко снижается при опухолях размером менее 5 мм, а также при стелющемся и инфильтративном характере роста опухоли, при фиксированном внутрипузырном сгустке, при отеке стенки пузыря. По эффективности выявления и оценке распространенности опухоли мочевого пузыря метод УЗИ уступает магнитно-резонансной (МРТ) и компьютерной томографии с контрастным усилением [106].

Компьютерная, магнитно-резонансная и позитронно-эмиссионная томография

Рекомендуется в качестве первичной диагностики всем пациентам с подозрением на РМП проведение магнитно-резонансной томографии (МРТ) органов малого таза [110, 111].

Уровень убедительности рекомендаций – B (уровень достоверности доказательств – 1).

Рекомендуется в качестве первичной и уточняющей диагностики всем пациентам с подозрением на РМП проведение обследования потенциальных зон распространения опухоли (лимфатические узлы, печень, кости, верхние отделы мочевого тракта и легкие):

- МРТ малого таза (согласно рекомендаций VI-RADS) и брюшной полости с контрастированием (как альтернатива – КТ брюшной полости и малого таза с контрастированием). При отсутствии признаков местного и регионарного поражения по данным МРТ (КТ) расширение объема исследования не требуется, рекомендуется проведение рентгенографии органов грудной клетки;

-  МРТ малого таза (согласно рекомендаций VI-RADS) и брюшной полости с контрастированием (как альтернатива – КТ брюшной полости и малого таза с контрастированием). При получении сведений за регионарное распространение по данным МРТ (КТ) требуется расширение объема исследования - КТ грудной полости [107-112].

- Проведение МРТ головного мозга целесообразно только при наличии неврологической симптоматики, указывающей на метастазирование в ткань и оболочки головной мозг [112, 113].

- Позитронно-эмиссионная томография всего тела, совмещенная с КТ (ПЭТ/КТ), может использоваться как дополнительный метод обследования при подозрении на отдаленные метастазы при стадии Т>2. ПЭТ/КТ не используется для Т-стадирования и определения распространённости опухоли по мочевыводящему тракту [114]

Уровень убедительности рекомендаций – С (уровень достоверности доказательств – 5).

Комментарии: Во всех случаях пациентам с подозрением на РМП целесообразно начинать исследование с МРТ малого таза по специальной методике (рекомендации VI-RADS) для оценки вероятности мышечной инвазии согласно критериев VI-RADS (как менее информативная альтернатива – КТ с контрастным усилением). Лучевые исследования лучше проводить до цистоскопии и внутрипузырных манипуляций, т.к. информативность в первые 7 дней после этого снижается [107, 108]. При этом следует понимать, что метод МРТ ограничен в визуализации опухолей Та-Т1 и выявлении признаков микроскопической инвазии за пределы стенки (Т3а) [109].

Следует понимать, что при правильно проведенном МРТ-сканировании (согласно рекомендациям VI-RADS) возможно визуализировать слои стенки мочевого пузыря, а также четко дифференцировать границы и структуру других органов малого таза в отличии от КТ. Поэтому МРТ обладает высокой чувствительностью и специфичностью в определении глубокой инвазии рака мочевого пузыря – около 90 \%. Особенно высокие показатели зарегистрированы на приборах с индукцией магнитного поля (напряженностью) 3,0 Тл [110, 111]. Оба метода (КТ и МРТ) примерно одинаково эффективны в оценке поражения лимфатических узлов и висцеральных очагов на основе критерия максимального поперечника (8 мм - для тазовых и 10 мм – для абдоминальных групп лимфатических узлов). Проведение МРТ головного мозга целесообразно только при наличии неврологической симптоматики, указывающей на метастазирование в ткань и оболочки головной мозг [112, 113].

Рекомендуется выполнять сцинтиграфию костей всего тела (остеосцинтиграфию) после установления диагноза РМП при подозрении на метастатическое поражение костей скелета вне зависимости от клинической стадии [115, 116].

Уровень убедительности рекомендаций – В (уровень достоверности доказательств – 2).
\subsection{Иные диагностические исследования}
\label{sec:}
Рекомендуется проводить диагностику и динамическое наблюдение пациентов с наследственными онкологическими синдромами с участием врача-генетика (медико-генетическое консультирование пробанда, а также его родственников – возможных носителей патогенной мутации) [313].

Уровень убедительности рекомендаций – С (уровень достоверности доказательств – 5).

Комментарии: наследственный РМП встречается в недифференцированной когорте пациентов с частотой около 0,5-1\% и относится, в основном, к проявлениям синдрома Линча – наследственного онкологического синдрома, который обусловлен мутацией в одном из генов системы репарации неспаренных нуклеотидов (MMR – mismatch repair deficiency): MLH1, MSH2, MSH6, PMS1, PMS2 или EPCAM. При этом синдроме в порядке убывания по частоте встречаемости описаны колоректальный рак, эндометриоидный рак, рак яичников, рак желудка, уротелиальные карциномы различных отделов мочевыделительной системы, рак предстательной железы и некоторые другие типы опухолей. РМП чаще развивается при мутации в MSH2. На наследственный характер заболевания может указывать молодой возраст пациента, неблагоприятный семейный онкологический анамнез, первично-множественные опухоли, в анамнезе пациента - новообразования в других органах-мишенях синдрома Линча. Молекулярно-генетическая диагностика заключается в анализе микросателлитной нестабильности, высокую степень которой (статус MSI-H) рассматривают как вероятное подтверждение заболевания. Минимальная панель из 5 мононуклеотидных STR-маркеров, зачастую используемая при колоректальном раке, в опухолях других типов обладает недостаточной чувствительностью. В связи с этим статус MSI-H в уротелиальных карциномах следует определять с помощью ИГХ-анализа и выявления потери экспрессии одного из ключевых участников системы репарации неспаренных оснований: MLH1, MSH2, MSH6 или PMS2 [312]. В случае выявления MSI-H наиболее информативным генетическим исследованием является определение герминальной мутации в генах-кандидатах синдрома Линча с помощью высокопроизводительного секвенирования (ВПС, англ. аналог – NGS, next generation sequencing) панели генов MMR. В отдельных случаях РМП может развиваться у носителей герминальных мутаций в генах BRCA1/2, MUTYH, RB1 и некоторых других. Если молодой пациент не удовлетворяет диагностическим критериям синдрома Линча, то ему может быть выполнено ВПС экзома или мультигенной онкологической панели [313].

Иные диагностические исследования могут понадобиться для дифференциальной диагностики РМП со следующими заболеваниями:

воспалительные заболевания мочевыводящих путей;

нефрогенная метаплазия;

аномалии развития мочевыделительного тракта;

плоскоклеточная метаплазия уротелия;

доброкачественные эпителиальные образования мочевого пузыря;

туберкулез;

сифилис;

эндометриоз;

хронический цистит;

метастазирование в мочевой пузырь меланомы, рака желудка и других опухолей (крайне редко).


