% \subsection{ Лучевая терапия}
% \label{sec:}
Воздействию лучевой терапии подлежат переходно-клеточные и плоскоклеточные опухоли.  Не показано проведение ЛТ при НМИ РМП. Лучевую терапию по радикальной программе применяют при тотальном поражении стенок мочевого пузыря. При НМИ РМП дистанционную ЛТ применяют с органосохраняющей целью при быстро рецидивирующих или обширных опухолях, при которых невозможна ТУР; при высоком риске прогрессии. Описаны положительные результаты применения ЛТ у пациентов с неудачами БЦЖ-терапии. В целом ЛТ при НМИ РМП применяют редко, рандомизированных сравнительных исследований с другими методами лечения нет.

3.4.1. Самостоятельная лучевая терапия
Рекомендуется химиолучевая терапия (предпочтительно)/самостоятельная ЛТ пациентам с МИ РМП c тяжелым соматическим статусом (ECOG≥ 2, Приложение Г1), которым не показано проведение радикальной ЦЭ [259, 260]

Уровень убедительности рекомендаций – С (уровень достоверности доказательств – 4).

Комментарии: лучевой терапии могут быть подвергнуты пациенты с нормальной функцией мочевого пузыря и достаточной его емкостью при отсутствии ИМТ (режим дозирования указан ниже по тексту) [260].

Рекомендуется пациентам с небольшими (менее 5 см) солитарными опухолями МП проведение брахитерапии для достижения ремиссии [261-263].

Уровень убедительности рекомендаций – С (уровень достоверности доказательств – 4).

Комментарии: несмотря на рекомендацию, в большинстве случаев проводят дистанционную ЛТ. 

Не рекомендуется использовать у пациентов с РМП подведенную суммарную очаговую дозу при ЛТ менее 60 Гр в связи с ее малой эффективностью [264].

Уровень убедительности рекомендаций – С (уровень достоверности доказательств – 4).

Комментарии: лучевая терапия по радикальной программе проводится в режиме фракционирования с разовой очаговой дозой (РОД) 2 Гр, 5 раз в неделю до суммарной очаговой дозы (СОД) 60-66 Гр непрерывным курсом. При этом, как правило, вначале в объем облучения включается весь таз (мочевой пузырь и зоны регионарного метастазирования) до СОД 44-46 Гр, затем МП и паравезикальная клетчатка 14-16 Гр (до СОД 60 Гр), затем – локально опухоль МП 6 Гр (до СОД 66 Гр). При Т2N0M0 в совокупности с G1-2 возможно проведение радиотерапии без включения в объем облучения на 1 этапе регионарных ЛУ. При наличии протонного комплекса целесообразно использовать энергию протонного пучка 70-250 МэВ. По данным разных авторов, 5-летняя выживаемость колеблется в пределах 24-46\%. При стадии Т2 5-летняя выживаемость составляет 25,3-59,0\%, при стадии Т3 – 9-38\% и при стадии Т4 – 0-16 \%. Ответ на проведенное лечение наблюдается у 35-70\% пациентов. Частота развития местных рецидивов составляет около 50\%. Осложнения возникают у 15\% пациентов; наиболее распространенные – цистит, гематурия, дизурические явления, проктит, диарея. Более чем у 2/3 мужчин развивается эректильная дисфункция.

Возможно проведение эскалации дозы на метастатически пораженные регионарные лимфатические узлы таза (при условии соблюдения границ толерантности со стороны здоровых тканей и органов).

При наличии технических возможностей, компетенции специалистов и клинического опыта возможно рассматривать вопрос проведения режима умеренного гипофракционирования до суммарной очаговой дозы 55 Гр за 20 сеансов [321].

3.4.2. Предоперационная лучевая терапия
Рекомендуется у пациентов с МИ РМП при проведении предоперационной ЛТ суммарная очаговая доза в пределах 20-45 Гр для снижения степени инвазии опухоли и предотвращения развития местного рецидива после хирургического вмешательства [265-267]

Уровень убедительности рекомендаций – С (уровень достоверности доказательств – 4).

Комментарии: в ряде проведенных исследований показано снижение числа местных рецидивов после предоперационной ЛТ, однако в других исследованиях не отмечено ее влияния на выживаемость и частоту местного рецидивирования.

3.4.3. Послеоперационная лучевая терапия
Рекомендуется проведение послеоперационной ЛТ у пациентов с МИ РМП при местно-распространенной стадии (рТ3–4) или R+ для профилактики рецидивирования [268-270].

Уровень убедительности рекомендаций – А (уровень достоверности доказательств – 2).

Комментарии: дистанционная радиотерапия проводится на область ложа удаленной опухоли в РОД 2 Гр, 5 раз в неделю до СОД 50 Гр, затем локально на остаточную опухоль РОД 2 Гр, 5 раз в неделю, СОД 10-16 Гр (СОД за оба этапа составит 60-66 Гр). При наличии метастатического поражения регионарных ЛУ на первом этапе ЛТ в объем облучения включаются регионарные лимфатические узлы мочевого пузыря, РОД 2 Гр, 5 раз в неделю, СОД 50 Гр, затем локально, определяемые по данным КТ метастатические лимфатические узлы РОД 2 Гр, 5 раз в неделю, СОД 16 Гр (СОД за оба этапа составит 66 Гр). В связи с изменением топографо-анатомических соотношений после удаления МП отмечают увеличение постлучевых осложнений, особенно со стороны желудочно-кишечного тракта.

3.4.4. Паллиативная лучевая терапия
С целью улучшения качества жизни рекомендуется проведение паллиативной лучевой терапии пациентам с РМП для купирования или уменьшения интенсивности симптомов первичной опухоли и/или метастазов. Режим фракционирования (включая режим умеренного гипофракционирования) определяется конкретной клинической ситуацией [322, 329-331].

Уровень убедительности рекомендаций – С (уровень достоверности доказательств – 2).

Комментарии: в рандомизированном клиническом исследовании, а также в нескольких сериях наблюдений было продемонстрировано симптоматическое улучшение после провденеия паллиативной лучевой терапии у пациентов с симптомами РМП [329-331]. В многоцентровом рандомизированном исследовании ВА09 (n 500), сравнивавшем эффективность и безопасность двух режимов паллиативной лучевой терапии (35Гр в 10 фракциях и 21 Гр в 3 фракциях), проводившейся для достижения симптоматического улучшения у пациентов с противопоказаниями к другим видам лечения РМП, не было выявлено значимых различий частоты снижения интенсивности проявлений заболевания в лечебных группах (71\% - в группе 35Гр и 64\% - в группе 21Гр) и доли пациентов с зарегистрированными нежелательными явлениями; общая выживаемость в группах исследования также не различалась (HR 0,99, 95\% CI 0,82-1,21, p 0,933) [331]. Основываясь на имеющихся данных, можно рекомендовать  применение нескольких режимов облучения с РОД 3-8 Гр до СОД 8-35 Гр. Допустима реализация режимов экстремального гипофракционирования, применяющихся при IG-IMRT (SBRT) и предусматривающих подведение РОД ≥7 Гр за несколько фракций, для достижения лучшего анальгезирующего эффекта [322]. Данный режим радиотерапии возможен только в специализированных центрах, обладающих соответствующим уровнем технического оснащения, подготовленным персоналом и клиническим опытом выполнения данной технологии.

