3.2.1. Радикальная цистэктомия
Радикальная ЦЭ является стандартным методом лечения, локализованного МИ РМП [143, 147]. Современное состояние проблемы все чаще требует более индивидуального подхода в лечении инвазивных и распространенных форм РМП. Оценка качества жизнь, работоспособность, ожидаемая продолжительность жизни, общее состояние пациента на момент операции – все это формирует новые тенденции в терапии, такие как комбинированные варианты химиолучевого лечения и органосохраняющей операции [148, 149].

Время от момента постановки диагноза до момента проведения операции точно не установлено, однако имеются данные, что выживаемость была выше в группе пациентов, которым выполнили операцию в течение 90 дней [150–152] (УД 2).

Рекомендуется выполнение радикальной ЦЭ пациентам группы высокого риска РМП при T2–4aN0M0 для достижения ремиссии [147]

Уровень убедительности рекомендаций – С (уровень достоверности доказательств – 4).

Комментарии: Показатели смертности ниже в центрах с большим опытом выполнения радикальной ЦЭ, общая 5-летняя выживаемость после ЦЭ составляет в среднем 40-60\% [153]:

− рТ1 – 75-83\%;

− рТ2 – 63-70\%;

− рТ3a – 47-53\%;

− рТ3b – 31-33\%;

− рТ4 – 19-28\%.

Рекомендуется выполнение радикальной ЦЭ пациентам, резистентным к химиолучевому лечению, при наличии свища, пациентам с тазовой болью, а также при рецидивирующей гематурии в качестве паллиативной помощи [154–156].

Уровень убедительности рекомендаций – С (уровень достоверности доказательств – 4).

Комментарии: общее количество осложнений ЦЭ составляет 9,7-30,0\%. Частота гнойно-септических осложнений достигает 0,28-30\%. Летальность после операции – 1,2-5,1\%. Интраоперационные осложнения достигают 5,3–9,7\%. Кровотечения составляют 3–7\%. Ранения прямой кишки при наличии лучевой терапии в анамнезе – 20-27\%, без лучевой терапии – 0,5-7,0 \%.

Наиболее распространенные послеоперационные осложнения [157]:

− лимфорея – 0-3 \%;

− кишечная непроходимость – 1-5 \%;

− желудочно-кишечные кровотечения – 1,5-2 \%;

− поздние послеоперационные осложнения в виде эректильной дисфункции – в 30-85 \% случаев;

− лимфоцеле – 0,1-2,6 \%;

− грыжи передней брюшной стенки – в 1,5-5,0 \% случаев.

Наличие только одного метастатического ЛУ (N1) не препятствует выполнению ортотопической пластики, но не в случае N2-3 [158].

У мужчин объем радикальной ЦЭ включает: удаление единым блоком (en bloc) мочевого пузыря с участком висцеральной брюшины и паравезикальной клетчаткой, предстательной железой и семенными пузырьками; тазовую (подвздошно-обтураторную) лимфаденэктомию. При опухолевом поражении простатической части уретры рекомендовано выполнение уретерэктомии [159, 160]. Также у мужчин возможно проведение нервосберегающей операции с сохранением кавернозных сосудисто-нервных пучков с целью профилактики развития эректильной дисфункции [159].

Женщинам рекомендован объем радикальной ЦЭ, включающий переднюю экзентерацию таза и двустороннюю тазовую лимфаденэктомию: удаление мочевого пузыря с участком висцеральной брюшины и паравезикальной клетчаткой, удаление матки с придатками, резекцию передней стенки влагалища [160].

Рекомендуется удаление регионарных лимфатических узлов в ходе выполнения радикальной ЦЭ. Выполнение расширенной лимфаденэктомии улучшает показатели выживаемости после радикальной ЦЭ по сравнению со стандартной методикой [161–165].

Уровень убедительности рекомендаций – B (уровень достоверности доказательств – 3).

Комментарии: объем тазовой лимфодиссекции включает в себя удаление ЛУ в области наружных и внутренних подвздошных сосудов, в обтураторной ямке, а также пресакральных ЛУ. Расширенная лимфодиссекция также подразумевает удаление ЛУ в области общих подвздошных сосудов до верхней границы – бифуркации аорты. Если краниальной границей служит нижняя брыжеечная артерия, то лимфодиссекция является суперрасширенной [161–165]. Оптимальный объем лимфаденэктомии не определен, однако преимущественное число рандомизированных исследований демонстрирует целесообразность выбора в пользу расширения границ лимфодиссекции как по показателям выживаемости без рецидива и прогрессии, так и по общей выживаемости [166–172].

Не рекомендуется при выполнении радикальной ЦЭ удаление уретры, которая может служить в дальнейшем для отведения мочи [173].

Уровень убедительности рекомендаций – С (уровень достоверности доказательств – 5).

Комментарии: Целесообразно сохранение уретры при отсутствии позитивного хирургического края.

3.2.1.1. Лапароскопическая и робот-ассистированная цистэктомия
Использование лапароскопической техники достаточно давно внедрено в практику и имеет большое количество публикаций, посвященных малоинвазивной методике. Эра робот-ассистированных операций – самая молодая среди всех существующих, однако число печатных работ по этой технологии конкурирует с таковыми по лапароскопии [174- 176]. Стоит отметить, что большинство представленных данных имеет низкий уровень доказательности – 4. По-видимому, это обусловлено некорректной стратификацией пациентов [174]. Лапароскопическая и робот-ассистированная ЦЭ рекомендованы к применению у пациентов с РМП, однако до сих пор остаются в фазе изучения. Лапароскопическая и робот-ассистированная техника могут применяться для лечения пациентов как с НМИ, так и с МИ РМП.

3.2.1.2. Варианты деривации мочи
Радикальная ЦЭ включает два непрерывных этапа: удаление мочевого пузыря с лимфодиссекцией и реконструктивно-пластический компонент. Вторым непрерывным этапом и является выбор способа деривации мочи [177]. Возраст >80 лет является противопоказанием к формированию резервуара [178].

Классификация видов деривации мочи:

− наружное отведение мочи (уретерокутанеостомия, кишечная пластика с формированием «сухих» и «влажных» стом);

− создание мочевых резервуаров, обеспечивающих возможность самостоятельного контролируемого мочеиспускания: орто- и гетеротопическая пластика мочевого пузыря;

− отведение мочи в непрерывный кишечник (уретеросигмостомия, операция Mainz-pouch II).

Рекомендуется при выборе способа деривации мочи подбирать метод, обеспечивающий пациенту высокий уровень качества жизни и наименьшее количество послеоперационных осложнений [177, 178, 292].

Уровень убедительности рекомендаций – С (уровень достоверности доказательств – 4).

Комментарии: тип отведения мочи не оказывает влияния на онкологические результаты. Не рекомендуется проведение лучевой терапии до оперативного вмешательства при выборе метода лечения с отведением мочи

Уретерокутанеостомия

У пациентов пожилого возраста или имеющих выраженные сопутствующие патологии предпочтительным методом является уретерокутанеостомия. Время операции, частота осложнений, пребывание в реанимации и длительность нахождения в стационаре ниже у пациентов после выведения мочеточников на кожу [179, 180]. При наружном отведении мочи пациенту необходимы мочеприемники.

Рекомендуется выполнять уретерокутанеостомию у пациентов с генерализованным или обширным местно-распространенным процессом при проведении ЦЭ с целью быстрого восстановления и проведения последующих этапов лечения [180].

Уровень убедительности рекомендаций – С (уровень достоверности доказательств – 4).

Комментарии: существует вероятность стеноза уретерокутанеостомы ввиду малого диаметра самой стомы.

Основные осложнения после операции:

− пиелонефрит;

− хроническая почечная недостаточность;

− стеноз устьев мочеточников (при формировании уретеро-уретероанастомоза «конец-в-бок»);

− стеноз стомы;

− кожные изменения вокруг стомы (мацерация, грибковое поражение).

Гетеротопический илеокондуит

Данный вариант формирования мочевого резервуара с выведением участка подвздошной кишки и формированием кутанеостомы является наиболее изученным и часто используемым. Тем не менее частота ранних послеоперационных осложнений достигает 48 \%. Пиелонефрит как наиболее частое осложнение наблюдается в 30–50 \% случаев [181].

Рекомендуется использовать илеоцекальный угол для гетеротопической пластики при операции типа Брикера для минимизации осложнений [181].

Уровень убедительности рекомендаций – С (уровень достоверности доказательств – 4).

Комментарии: наиболее часто встречающиеся осложнения [182–184]: 

− пиелонефрит;

− кишечная непроходимость;

− стеноз мочеточниково-резервуарных анастомозов;

− стеноз стомы;

− кожные изменения вокруг стомы (мацерация, грибковое поражение).

Гетеротопический илеокондуит («сухая» стома)

Рекомендуется пациентам для создания резервуара с «сухой» стомой формирование детубулярного резервуара из участка подвздошной кишки низкого давления с формированием стомы для самокатетеризации [185–190].

Уровень убедительности рекомендаций – С (уровень достоверности доказательств – 4).

Комментарии: Хорошее удерживание мочи в дневное и ночное время отмечено многими пациентами и достигает 90\% [188]. Стеноз аппендикулярной стомы встречается в 15-23\% случаев [189]. Выбор данного варианта реконструктивной пластики является достаточно трудоемким и требует навыка и опыта хирурга [190].

Ортотопический резервуар

Формирование ортотопического резервуара предполагает его расположение в полости таза, на месте удаленного МП, и создание резервуарно-уретрального анастомоза. Этот метод позволяет пациенту в дальнейшем самостоятельно контролировать акт мочеиспускания [147,190,191].

Рекомендуется выполнение ортотопической пластики каждому пациенту при отсутствии противопоказаний и вовлечения опухолью мочеиспускательного канала для улучшения качества жизни [147,190, 191,192].

Уровень убедительности рекомендаций – С (уровень достоверности доказательств – 4).

Комментарии: женщинам также возможно выполнение ортотопической пластики при условии тщательно изученной шейки мочевого пузыря (биопсия с целью выявления опухолевых участков) [192].

Рекомендуется использовать: подвздошную кишку, илеоцекальный угол, восходящую ободочную или сигмовидную кишку при формировании ортотопических мочевых резервуаров для минимизации осложнений [193, 194].

Уровень убедительности рекомендаций – С (уровень достоверности доказательств – 4).

Комментарии: противопоказания для операции – опухолевое поражение уретры ниже семенного бугорка; выраженная хроническая почечная недостаточность.

Наиболее частые осложнения [193]:

− дневное недержание мочи (5,4-30,0\%);

− ночное недержание мочи (18,6-39,0\%);

− пиелонефрит;

− метаболические осложнения (гиперхлоремический ацидоз);

− конкрементообразование;

− стриктура резервуарно-уретрального анастомоза.

3.2.2. Органосохраняющие операции
Органосохраняющее лечение мышечно-инвазивного рака мочевого пузыря направлено на сохранение пораженного органа и, как следствие, качества жизни пациентов без ухудшения выживаемости.

Рекомендуется проведение органосохраняющего лечения отобранным пациентам, соответствующим следующим критериям:

- солитарная опухоль мочевого пузыря, вне его шейки;

- категория рТ2a–b;

- грейд G1–2 или LG;

- отсутствие гидронефроза, обусловленного опухолью;

- хорошая функция мочевого пузыря до лечения;

- нормальный показатель ПСА (исследование общей и свободной фракции крови);

- отрицательный результат мультифокальной биопсии предстательной железы (опционально);

- отсутствие в анамнезе указаний на резекцию мочевого пузыря, или чреспузырную аденомэктомию, или чреспузырное удаление конкрементов мочевого пузыря;

- отсутствие в анамнезе указаний на лучевую терапию на область малого таза;

- отсутствие протяженных стриктур мочеиспускательного канала;

- противопоказания к РЦЭ [195-198].

Уровень убедительности рекомендаций – В (уровень достоверности доказательств – 3).

Не рекомендуется использование только хирургического лечения, только ХТ или только ЛТ в качестве самостоятельных методов органосохраняющего лечения МИ РМП [195,198,199].

Уровень убедительности рекомендаций – В (уровень достоверности доказательств – 3).

Комментарии: только ТУР мочевого пузыря, только ХТ или только ЛТ существенно уступают радикальной цистэктомии с НХТ или АХТ в отношении онкологических результатов, в связи с чем не рекомендуются к использованию в широкой клинической практике [148,149].

Рекомендуется использование трехмодального лечения, включающего максимальную ТУР мочевого пузыря с последующим проведением химиолучевой терапии, для сохранения мочевого пузыря отобранным пациентам с МИ РМП, соответствующих критериям, перечисленным выше [195,198,199].

Уровень убедительности рекомендаций – В (уровень достоверности доказательств – 3).

Комментарии: наиболее эффективным методом органосохраняющего лечения, который может использоваться у тщательно отобранных больных, является трехмодальная терапия, подразумевающая выполнение максимальной ТУР мочевого пузыря с последующим проведением химио-лучевой терапии (ХЛТ). Обоснованием сочетания ТУР с ЛТ является необходимость достичь полного локального контроля над первичной опухолью и регионарными лимфатическими коллекторами. Введение в схему лечения радиосенсибилизирующих цитостатиков (cенсибилизирующих препаратов, используемых для фотодинамической/лучевой терапии) направлено на усиление эффекта облучения, а также потенциально способно элиминировать микрометастазы.

В составе трехмодального лечения описаны разные схемы ХТ, включая монотерапию цисплатином** [314], а также монотерапию гемцитабином** [315]. В рандомизированном исследовании II фазы два ежедневных сеанса облучения с комбинированной ХТ (фторурацил** и цисплатин**) и один сеанс ежедневного облучения с монотерапией гемцитабином** продемонстрировали сопоставимую 3-летнюю выживаемость без отдаленных метастазов (78\% и 84\% соответственно) при большей частоте гематологических НЯ 4 степени тяжести в группе полихимиотерапии [316].

Пятилетняя специфическая и общая выживаемость больных, подвергнутых трехмодальной терапии, колеблется от 50\% до 82\% и от 36\% до 74\%, соответственно [314, 316]. Большинство рецидивов рака мочевого пузыря не инвазирует детрузор и может быть излечено консервативно. Спасительная цистэктомия требуется примерно у 10-15\% пациентов, получавших трехмодальное лечение. Отдаленные результаты спасительных операций сопоставимы с результатами первичных радикальных цистэктомий, хотя частота осложнений у облученных пациентов выше [317].

Рандомизированных исследований, сравнивающих радикалную ЦЭ и трехмодальное лечение, не проводилось. Cистематический обзор, включивший данные более 30 000 пациентов из 57 исследований, не выявил достоверных различий выживаемости между больными, подвергнутыми радикальной ЦЭ и трехмодальной терапии. Однако при сроке наблюдения 10 лет специфическая и общая выживаемость оказались выше у пациентов с сохраненным мочевым пузырем [318].  Профиль безопасности трехмодальной терапии благоприятный. Комбинированный анализ данных пациентов, входивших в 4 исследования RTOG, показал, что при медиане наблюдения 5,4 года частота поздней гастроинтестинальной и мочевой токсичности 3 степени тяжести составляет 1,9\% и 5,7\% соответственно; нежелательных явлений 4 степени тяжести не зарегистрировано [319]. Ретроспективные данные показали преимущество качества жизни пациентов, подвергнутых трехмодальной терапии, по сравнению с больными, перенесшими РЦЭ [86].

Наиболее часто используемые режимы химиолучевой терапии в составе трехмодального лечения приведены в таблице 2 [314-316, 323].
